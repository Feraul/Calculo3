\documentclass[oneside,a4paper,12pt]{article}
\usepackage[english,brazilian]{babel}
\usepackage[alf]{abntex2cite}
\usepackage[utf8]{inputenc}
\usepackage[T1]{fontenc}

\usepackage{lastpage}			  % Usado pela Ficha catalográfica
\usepackage{indentfirst}		  % Indenta o primeiro parágrafo de cada
\usepackage[top=20mm, bottom=20mm, left=20mm, right=20mm]{geometry}
\usepackage{framed}
\usepackage{booktabs}

\usepackage{float}
\usepackage{color}				  % Controle das cores
\usepackage{graphicx}			  % Inclusão de gráficos
\usepackage{microtype} 		      % para melhorias de justificação
\usepackage{booktabs}
\usepackage{multirow}
\usepackage[table]{xcolor}
\usepackage{subfig}
\usepackage{epstopdf}
\usepackage{hyperref}

\usepackage[mathcal]{eucal}
\usepackage{amsmath}               % great math stuff
\usepackage{amsfonts}              % for blackboard bold, etc
\usepackage{amsthm}                % better theorem environments
\usepackage{amssymb}
\usepackage{mathrsfs}
\DeclareMathAlphabet{\mathpzc}{OT1}{pzc}{m}{it}
\usepackage{undertilde}            % botar tilde embaixo da letra
\usepackage{mathptmx}          % fonte
\usepackage{graphicx}
\graphicspath{{./Figuras/}}    
\definecolor{shadecolor}{rgb}{0.8,0.8,0.8}


%FAZ EDICOES AQUI (somente no conteudo que esta entre entre as ultimas  chaves de cada linha!!!)
\newcommand{\universidade}{Universidade Federal de Pernambuco}
\newcommand{\centro}{Centro Acadêmico do Agreste}
\newcommand{\departamento}{Núcleo de Tecnologia}
\newcommand{\curso}{Engenharia de Produção}
\newcommand{\professor}{Fernando R. L. Contreras}
\newcommand{\disciplina}{Cálculo Diferencial e Integral 3}
%ATE AQUI !!!

\begin{document}
	\pagestyle{empty}
	
	\begin{center}
	%\includegraphics[width=\linewidth/6]{logoUFPE.jpg}%LOGOTIPO DA INSTITUICAO
	 	\vspace{0pt}
	 	
		\universidade
		\par
		\centro
		\par
		\departamento
		\par
		\curso
		\par
		\vspace{08pt}
		\text{ Prova 1 - Cálculo Diferencial e Integral 3}\\
		\text{Prof. Fernando R. L. Contreras}	
	\end{center}
	
	%\vspace{0.5pt}
	
	\begin{flushleft}
		\textbf{Aluno(a):}
	\end{flushleft}
	
\begin{itemize}
\item[1.] Método de Newton. As sequências vêm da formula recursiva para o método de Newton, $x_{n+1}=x_{n}-\frac{f(x_{n})}{f'(x_{n})}$. A sequência converge? Em caso afirmativo, para qual valor? Identifique a função $f$ que gera a sequência $x_{0}=1, x_{n+1}=x_{n}-\frac{x_{n}^{2}-2}{2x_{n}}=\frac{x_{n}}{2}+\frac{1}{x_{n}}$.
\end{itemize}
\begin{itemize}
\item[2.] Determine o polinômio de Taylor de ordem 2 da função dada, em volta do ponto $(1,1)$ dado por  $f(x,y)=x^{3}+2x^{2}y+3y^{3}+x-y$. 
 \end{itemize}
 \begin{itemize}
\item [3.] A série $\sum_{n=1}^{\infty} (-1)^{n}\frac{n+2}{n(n+1)} $ é absolutamente convergente ou condicionalmente convergente? Justifique sua resposta.
\end{itemize}
\begin{itemize}
\item[4.] Determine o intervalo de convergência da série de potencia  $\sum_{n=1}^{\infty}\frac{x^{n}}{2+n^{2}} $ .
\end{itemize}
\begin{itemize}
	\item[Opcional.] Escreva a definição de limite de sequência na forma simbólica.
\end{itemize}

	\flushbottom
	\flushright
     \begin{center}
     	%\includegraphics[width=\linewidth/6]{logoUFPE.jpg}%LOGOTIPO DA INSTITUICAO
     	\vspace{50pt}
     	
     	\universidade
     	\par
     	\centro
     	\par
     	\departamento
     	\par
     	\curso
     	\par
     	\vspace{08pt}
     	\text{ Prova 1 - Cálculo Diferencial e Integral 3}\\
     	\text{Prof. Fernando R. L. Contreras}	
     \end{center}
     
     \begin{flushleft}
     	\textbf{Aluno(a):}
     \end{flushleft}
 
 \begin{itemize}
 	\item[1.] Método de Newton. As sequências vêm da formula recursiva para o método de Newton, $x_{n+1}=x_{n}-\frac{f(x_{n})}{f'(x_{n})}$. A sequência converge? Em caso afirmativo, para qual valor? Identifique a função $f$ que gera a sequência $x_{0}=1, x_{n+1}=x_{n}-\frac{x_{n}^{2}-2}{2x_{n}}=\frac{x_{n}}{2}+\frac{1}{x_{n}}$.
 \end{itemize}
 \begin{itemize}
 	\item[2.] Determine o polinômio de Taylor de ordem 2 da função dada, em volta do ponto $(1,1)$ dado por  $f(x,y)=x^{3}+2x^{2}y+3y^{3}+x-y$. 
 \end{itemize}
 \begin{itemize}
 	\item [3.] A série $\sum_{n=1}^{\infty}(-1)^{n}\frac{n+2}{n(n+1)} $ é absolutamente convergente ou condicionalmente convergente? Justifique sua resposta.
 \end{itemize}
 \begin{itemize}
 	\item[4.] Determine o intervalo de convergência da série de potencia  $\sum_{n=1}^{\infty} \frac{x^{n}}{2+n^{2}} $ .
 \end{itemize}
 \begin{itemize}
 	\item[Opcional.] Escreva a definição de limite de sequência na forma simbólica.
 \end{itemize}
\flushbottom
\flushright
\end{document}

