
%%%%%%%%%%%%%%%%%%%%%%%%%%%%%%%%%%%%%%%%%%%%%%%%%%%%%%%%%%%%%%%%%%%%%%%%
%    Option test file, will be created during the first LaTeX run:
\begin{filecontents}{exercise.thm}
\def\th@exercise{%
  \normalfont % body font
  \thm@headpunct{:}%
}
\end{filecontents}
%%%%%%%%%%%%%%%%%%%%%%%%%%%%%%%%%%%%%%%%%%%%%%%%%%%%%%%%%%%%%%%%%%%%%%%%

\documentclass[12pt,openright,oneside,a4paper,english,french,spanish,brazil]{article}
% ---
% Pacotes básicos 
% ---
\usepackage{lmodern}			  % Usa a fonte Latin Modern			
\usepackage[T1]{fontenc}		  % Selecao de codigos de fonte.
\usepackage[utf8]{inputenc}	      % Codificacao do documento (conversão automática dos acentos)
\usepackage[top=20mm, bottom=20mm, left=20mm, right=20mm]{geometry}
\usepackage{lastpage}			  % Usado pela Ficha catalográfica
\usepackage{indentfirst}		  % Indenta o primeiro parágrafo de cada seção.
\usepackage{color}				  % Controle das cores
\usepackage{graphicx}			  % Inclusão de gráficos
\usepackage{microtype} 		      % para melhorias de justificação
\usepackage{booktabs}
\usepackage{multirow}
\usepackage[table]{xcolor}
\usepackage{subfig}
\usepackage{epstopdf}
\usepackage{hyperref}
\usepackage[mathcal]{eucal}
\usepackage{amsmath}               % great math stuff
\usepackage{amsfonts}              % for blackboard bold, etc
\usepackage{amsthm}                % better theorem environments
\usepackage{amssymb}
\usepackage{mathrsfs}
\DeclareMathAlphabet{\mathpzc}{OT1}{pzc}{m}{it}
\usepackage{undertilde}            % botar tilde embaixo da letra
\usepackage{mathptmx}          % fonte
\usepackage{latexsym}
\usepackage{makeidx}            % para definir o índice
\usepackage{epsfig}             % para introduzir figuras no formato eps
\usepackage{graphicx,color}     % permite a inclusao de figuras
\usepackage{verbatim}
\usepackage{gensymb}
\usepackage{titling}
\newcommand{\subtitle}[1]{%
	\posttitle{%
		\par\end{center}
	\begin{center}\Large#1\end{center}
	\vskip0.5em}%
}



\newtheorem{df}{Definição}
\newtheorem{ex}{Exemplo}
\newtheorem{teo}{Teorema}

\newtheoremstyle{note}% name
  {3pt}%      Space above
  {3pt}%      Space below
  {}%         Body font
  {}%         Indent amount (empty = no indent, \parindent = para indent)
  {\itshape}% Thm head font
  {:}%        Punctuation after thm head
  {.5em}%     Space after thm head: " " = normal interword space;
        %       \newline = linebreak
  {}%         Thm head spec (can be left empty, meaning `normal')

\theoremstyle{note}
\newtheorem{note}{Note}

\newtheoremstyle{citing}% name
  {3pt}%      Space above, empty = `usual value'
  {3pt}%      Space below
  {\itshape}% Body font
  {}%         Indent amount (empty = no indent, \parindent = para indent)
  {\bfseries}% Thm head font
  {.}%        Punctuation after thm head
  {.5em}%     Space after thm head: " " = normal interword space;
        %       \newline = linebreak
  {\thmnote{#3}}% Thm head spec

\theoremstyle{citing}
\newtheorem*{varthm}{}% all text supplied in the note

\newtheoremstyle{break}% name
  {9pt}%      Space above, empty = `usual value'
  {9pt}%      Space below
  {\itshape}% Body font
  {}%         Indent amount (empty = no indent, \parindent = para indent)
  {\bfseries}% Thm head font
  {.}%        Punctuation after thm head
  {\newline}% Space after thm head: \newline = linebreak
  {}%         Thm head spec

\theoremstyle{break}
\newtheorem{bthm}{B-Theorem}

\theoremstyle{exercise}
\newtheorem{exer}{Exercise}

\swapnumbers
\theoremstyle{plain}
\newtheorem{thmsw}{Theorem}[section]
%\newtheorem{corsw}[thm]{Corollary}
\newtheorem{propsw}{Proposition}
%\newtheorem{lemsw}[thm]{Lemma}

%    Because the amsmath pkg is not used, we need to define a couple of
%    commands in more primitive terms.
\let\lvert=|\let\rvert=|
\newcommand{\Ric}{\mathop{\mathrm{Ric}}\nolimits}

%    Dispel annoying problem of slightly overlong lines:
\addtolength{\textwidth}{8pt}

\title{ \textbf{Notas de Aula}}

\author{\textbf{Fernando Contreras}\\
	\large Nucleo de Tecnologia\\
	Universidade Federal de Pernambuco (UFPE)}



\begin{document}
	\begin{center}
		Universidade Federal de Pernambuco (UFPE)\\
		Centro Acadêmico do Agreste\\
		Núcleo de Tecnologia\\
		
		Lista 1 de Calculo Diferencial e Integral 3\\
		Prof. Fernando RL Contreras
	\end{center}


Sejam os seguintes problemas

%\begin{multicols}{2}

\begin{itemize}
	\item[1.] Determine $\lim\limits_{n\rightarrow \infty} (\frac{1}{n})^{1/\ln n} $ \quad \quad Rpta. $e^{-1}$
\end{itemize}
\begin{itemize}
	\item[2.] Calcule $\lim\limits_{n\rightarrow \infty} \frac{(-1)^{n}n^{3}}{n^{3}+2n^{2}+1}$ \quad \quad Rpta. Diverge
\end{itemize}
\begin{itemize}
	\item [3.] Determine $\lim\limits_{n\rightarrow \infty} \frac{1}{\sqrt{n^{2}-1}-\sqrt{n^{2}+n}}$ \quad \quad Rpta. -2
\end{itemize}
\begin{itemize}
	\item[4.] Calcule $\lim\limits_{n\rightarrow \infty} \frac{1.3.5...(2n-1)}{(2n)^{n}}$ \quad \quad Rpta. 0
\end{itemize}
\begin{itemize}
	\item[5.] Determinar $\lim\limits_{n\rightarrow } \sinh (\ln (n))$ \quad \quad Rpta. Diverge
\end{itemize}
%\end{multicols}
Nos seguintes problemas determine se a sequência converge ou não e encontre o limite se ele converge. 
%\begin{multicols}{2}

\begin{itemize}
	\item[1.] $a_{n}=\frac{1.2.3....(2n-1)}{n!}$ \quad \quad Rpta. Diverge
\end{itemize}
\begin{itemize}
	\item[2.] $a_{n}=1-(0.2)^{n}$ \quad \quad Rpta. Converge a 1
\end{itemize}
\begin{itemize}
	\item[3.]  $a_{n}=\frac{3+5n^{2}}{n+n^{2}}$ \quad \quad Rpta. Converge a 0
\end{itemize}
\begin{itemize}
	\item[4.] Calcule o limite da sequência $\left(  \sqrt{2},\sqrt{2\sqrt{2}},\sqrt{2\sqrt{2\sqrt{2}}}, ...\right)  $ \quad \quad Rpta.2
\end{itemize}
\begin{itemize}
	\item[5.]$a_{n}=\frac{n^{2}}{2n-1}\sin (\frac{1}{n})$ \quad \quad Rpta. Converge a 1/2
\end{itemize}
\begin{itemize}
	\item[6.] $\left(  \arctan(2n)\right)  $ \quad \quad Rpta. Converge a $\frac{\pi}{2}$
\end{itemize}
\begin{itemize}
	\item [7.] $\left(  \frac{\ln(n)}{\ln(2n)}\right) $ \quad \quad Rpta. Converge a 1
\end{itemize}
\begin{itemize}
	\item [8.] $\left(  n^{2}e^{-n}\right) $ \quad \quad Rpta. Converge a 0
\end{itemize}
\begin{itemize}
	\item [9.] $a_{n}=\frac{(-3)^{n}}{n!}$ \quad \quad Rpta. Converge a 0
\end{itemize}
\begin{itemize}
	\item [10.] $a_{n}=\frac{(\ln n)^{2}}{n}$ \quad \quad Rpta. Converge a 0
\end{itemize}
\begin{itemize}
	\item [11.] $a_{n}=\sqrt{\frac{n+1}{9n+1}}$ \quad\quad Rpta. Converge a 1/3
\end{itemize}
\begin{itemize}
	\item [12.] $a_{n}=\frac{1}{n}\int_{1}^{n}\frac{1}{x}dx$ \quad\quad Rpta. Converge a 0
\end{itemize}
\begin{itemize}
	\item [13.]$a_{n}=\frac{(\ln n)^{200}}{n}$ \quad\quad Rpta. Converge a 0
\end{itemize}
\begin{itemize}
	\item [14.] $a_{n}=\frac{n!}{n^{n}}$ (Sugestão compare com $1/n$). \quad\quad Rpta. Converge a 0
\end{itemize}
\begin{itemize}
	\item [15.] $a_{n}=(\frac{x^{n}}{2n+1})^{1/n}$, se $x>0$ \quad\quad Rpta. Converge a 0
\end{itemize}
\begin{itemize}
	\item [16.] $a_{n}=\frac{(-1)^{n+1}}{2n-1}$ \quad\quad  Rpta. Converge a 0
\end{itemize}
\begin{itemize}
	\item [17.] $a_{n}=(3^{n}+5^{n})^{1/n}$  \quad\quad  Rpta. Converge a 5
\end{itemize}
%\end{multicols} 
Assuma que cada sequência convirja e encontre o limite.
%\begin{multicols}{2}
\begin{itemize}
	\item [17.] $a_{1}=2$, $a_{n+1}= \frac{72}{1+a_{n}}$ \quad\quad  Rpta. 8
\end{itemize}
\begin{itemize}
	\item [18.] $a_{1}=-1$, $a_{n+1}= \frac{a_{n}+6}{a_{n}+2}$ \quad\quad  Rpta. 2
\end{itemize}	
\begin{itemize}
	\item [19.] $a_{1}=-4$, $a_{n+1}= \sqrt{8+2a_{n}}$ \quad\quad  Rpta. 4
\end{itemize}
\begin{itemize}
	\item [20.] $a_{1}=0$, $a_{n+1}= \sqrt{8+2a_{n}}$ \quad\quad  Rpta. 4
\end{itemize}
\begin{itemize}
	\item [21.] $a_{1}=5$, $a_{n+1}= \sqrt{5a_{n}}$ \quad\quad  Rpta. 5
\end{itemize}
\begin{itemize}
	\item [22.] $a_{1}=3$, $a_{n+1}= 12-\sqrt{a_{n}}$ \quad\quad  Rpta. 9
\end{itemize}
\begin{itemize}
	\item [23.] $2,2+\frac{1}{2}, 2+\frac{1}{2+\frac{1}{2}}, 2+\frac{1}{2+\frac{1}{2+\frac{1}{2}}},...$ \quad\quad  Rpta. $1+\sqrt{2}$
\end{itemize}
\begin{itemize}
	\item [24.] $\sqrt{1}, \sqrt{1+\sqrt{1}}, \sqrt{1+\sqrt{1+\sqrt{1}}}, \sqrt{1+\sqrt{1+\sqrt{1}+\sqrt{1}}}, ...$ \quad\quad  Rpta. $\frac{1+\sqrt{5}}{2}$
\end{itemize}
\begin{itemize}
	\item [25.] Método de Newton. As sequências vêm da formula recursiva para o método de Newton, $x_{n+1}=x_{n}-\frac{f(x_{n})}{f'(x_{n})}$. A sequência converge? Em caso afirmativo, para qual valor? Identifique a função $f$ que gera a sequência $x_{0}=1, x_{n+1}=x_{n}-\frac{x_{n}^{2}-2}{2x_{n}}=\frac{x_{n}}{2}+\frac{1}{x_{n}}$  \quad\quad  Rpta. $\sqrt{2}$ 
\end{itemize}
\begin{itemize}
	\item [26.] O primeiro termo de uma sequência é $x_{1}=1$. Cada um dos termos seguintes é a soma de todos os seus antecedentes: $x_{n+1}=x_{1}+x_{2}+...+x_{n}$. Escreva os primeiros termos da sequência suficientes para deduzir uma fórmula geral para $x$, que seja verdadeira para $n\geq2$ 
\end{itemize}
\begin{itemize}
	\item [27.]\textbf{(a)} Presumindo que $\lim\limits_{n\rightarrow \infty}(\frac{1}{n^{c}})=0$ se $c$ for qualquer constante
	positiva, mostre que $\lim\limits_{n\rightarrow \infty}(\frac{\ln (n)}{n^{c}})=0$ se $c$ for qualquer constante positiva.\\
	\textbf{ (b)} Prove que $\lim\limits_{n\rightarrow \infty}(\frac{1}{n^{c}})=0$ se $c$ for qualquer constante positiva. (Sugestão: se $\epsilon =0.001$ e $c=0.04$, quão grande deve ser $N$ para assegurar que $\left| 1/n^{c} -0 \right| < \epsilon$ e $n>N$ ? ) 
\end{itemize}
%\end{multicols} 
%\begin{multicols}{2}
Determine se a sequência é monotônica e
se é limitada.
\begin{itemize}
	\item [28.] $a_{n}=\frac{3n+1}{n+1}$
\end{itemize}
\begin{itemize}
	\item [29.] $a_{n}=\frac{(2n+3)!}{(n+1)!}$
\end{itemize}
\begin{itemize}
	\item [30.] $a_{n}=\frac{2^{n}3^{n}}{n!}$
\end{itemize}
\begin{itemize}
	\item [31.] $a_{n}=2-\frac{2}{n}-\frac{1}{2^{n}}$
\end{itemize}
\begin{itemize}
	\item [32.] Suponha que $f(x)$ seja derivável para todo $x$ em [0,1] e $f(0)=0$. Defina a sequência $\left( a_{n}\right) $ pela regra $a_{n}=nf(1/n)$. Mostre que $\lim\limits_{n\rightarrow \infty} a_{n}=f'(0)$. Utilize o resultado do problema (32) para encontrar os limites das sequências dadas em 33. e 34.
\end{itemize}
\begin{itemize}
	\item [33.] $a_{n}=n(e^{1/n}-1)$ \quad\quad  Rpta. 1 
\end{itemize}
\begin{itemize}
	\item [34.] $a_{n}=n\ln (1+\frac{1}{n})$ \quad\quad  Rpta. 2
\end{itemize}
\begin{itemize}
	\item [35.] Mostre que a sequência definida por $a_{1}=2$ e $a_{n+1}=\frac{1}{3-a_{n}}$ satisfaz $0<a_{n}\leq 2$ e é decrescente. Deduza que a sequência é convergente e encontre o seu limite.
\end{itemize}
\begin{itemize}
	\item [36.] Demostre que se $\lim\limits_{n\rightarrow \infty} a_{n}=0$ e $\left(  b_{n} \right)  $ for limitada, então $\lim\limits_{n\rightarrow \infty} a_{n}b_{n}=0$. 
\end{itemize}
\begin{itemize}
	\item [37.] Suponha que $\left\lbrace a_{n}\right\rbrace $ é uma sequência crescente não limitada. Mostre que  $\lim\limits_{n\rightarrow \infty} a_{n}=+\infty$.
\end{itemize}
\begin{itemize}
	\item [38.] Suponha que $A>0$. Dado $x_{1}$ arbitrário, defina a sequência $\left\lbrace x_{n} \right\rbrace$ de maneira recursiva como segue: $x_{n+1}=\frac{1}{2}(x_{n}+\frac{A}{x_{n}})$ se $n\geq 1$. Demostre que se $L=\lim\limits_{n\rightarrow \infty} x_{n}$ existe, então $L=\pm \sqrt{A}$.
\end{itemize}
\begin{itemize}
	\item [39.] Considere a sequência $\left\lbrace a_{n}\right\rbrace $ definida de maneira recursiva por $a_{1}=2$; $a_{n+1}=\frac{1}{2}(a_{n}+4)$ para $n\geq 1$ (a) Demostrar mediante indução sobre $n$ que $a_{n}<4$ para cada $n$ e que $(a_{n})$ é uma sequência crescente. (b) Determine o limite desta sequência.
\end{itemize}
\begin{itemize}
	\item [40.] O tamanho da população de peixes pode ser modelado pela formula $p_{n+1}=\frac{bp_{n}}{a+p_{n}}$ onde $p_{n}$ é o tamanho da população de peixes depois de $n$ anos e $a$ e $b$ são constantes positivas que dependem da espécie e de seu habitat. Suponha que a população no ano 0 seja $p_{0}>0$ (a) Moesstre que se $(p_{n})$ é convergente, então os únicos valores possíveis para seu limite são $0$ e $b-a$. (b) mostre que $p_{n+1}<(b/a)p_{n}$.
\end{itemize}























\end{document}
