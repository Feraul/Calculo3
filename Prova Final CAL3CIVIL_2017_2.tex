\documentclass[oneside,a4paper,12pt]{article}
\usepackage[english,brazilian]{babel}
\usepackage[alf]{abntex2cite}
\usepackage[utf8]{inputenc}
\usepackage[T1]{fontenc}

\usepackage{lastpage}			  % Usado pela Ficha catalográfica
\usepackage{indentfirst}		  % Indenta o primeiro parágrafo de cada
\usepackage[top=20mm, bottom=20mm, left=20mm, right=20mm]{geometry}
\usepackage{framed}
\usepackage{booktabs}

\usepackage{float}
\usepackage{color}				  % Controle das cores
\usepackage{graphicx}			  % Inclusão de gráficos
\usepackage{microtype} 		      % para melhorias de justificação
\usepackage{booktabs}
\usepackage{multirow}
\usepackage[table]{xcolor}
\usepackage{subfig}
\usepackage{epstopdf}
\usepackage{hyperref}

\usepackage[mathcal]{eucal}
\usepackage{amsmath}               % great math stuff
\usepackage{amsfonts}              % for blackboard bold, etc
\usepackage{amsthm}                % better theorem environments
\usepackage{amssymb}
\usepackage{mathrsfs}
\DeclareMathAlphabet{\mathpzc}{OT1}{pzc}{m}{it}
\usepackage{undertilde}            % botar tilde embaixo da letra
\usepackage{mathptmx}          % fonte
\usepackage{graphicx}
\graphicspath{{./Figuras/}}    
\definecolor{shadecolor}{rgb}{0.8,0.8,0.8}


%FAZ EDICOES AQUI (somente no conteudo que esta entre entre as ultimas  chaves de cada linha!!!)
\newcommand{\universidade}{Universidade Federal de Pernambuco}
\newcommand{\centro}{Centro Acadêmico do Agreste}
\newcommand{\departamento}{Núcleo de Tecnologia}
\newcommand{\curso}{Engenharia Civil}
\newcommand{\professor}{Fernando R. L. Contreras}
\newcommand{\disciplina}{Cálculo Diferencial e Integral 3}
%ATE AQUI !!!

\begin{document}
	\pagestyle{empty}
	
	\begin{center}
	%\includegraphics[width=\linewidth/6]{logoUFPE.jpg}%LOGOTIPO DA INSTITUICAO
	 	\vspace{0pt}
	 	
		\universidade
		\par
		\centro
		\par
		\departamento
		\par
		\curso
		\par
		\vspace{08pt}
		\text{ Prova Final - Cálculo Diferencial e Integral 3}\\
		\text{Prof. Fernando R. L. Contreras}	
	\end{center}
	
	%\vspace{0.5pt}
	
	\begin{flushleft}
		Aluno(a):
	\end{flushleft}
	
\begin{itemize}
\item[1.] Seja $\vec{E}(x,y,z)=\frac{q}{x^{2}+y^{2}+z^{2}}\frac{1}{\sqrt{x^{2}+y^{2}+z^{2}}}(x,y,z)$ o campo elétrico criado por uma carga $q$ localizada na origem. Calcule o fluxo de $\vec{E}$ através da superfície esférica de raio $r$ e centrada na origem, com normal $\vec{n}$ apontando para fora da esfera. Considere, $A(S)=4\pi r^{2}$.
\end{itemize}
\begin{itemize}
\item[2.] Seja $\vec{F}$ um campo de classe $C^{1}$ num aberto contendo a fronteira do cubo $0\leq x \leq 1$, $0\leq y \leq1$ e $0\leq z \leq1$. Seja $\vec{n}$ a normal apontando para fora do cubo. Mostre que $\mathlarger{\iint}_{S}rot(\vec{F})\cdot \vec{n}ds=0$ 
 \end{itemize}
 \begin{itemize}
\item [3.] Calcule $\mathlarger{\iint}_{S}rot(F)dS$, onde $S=\left\lbrace  (x,y,z)\in \mathbb{R}^{3}\quad \text{tal que}\quad x=-1+y^{2}+z^{2},\quad x\leq 0 \right\rbrace $ e o campo $\vec{F}$ é definido por $\vec{F}(x,y,z)=(xz,ze^{x},-y)$.
\end{itemize}
\begin{itemize}
\item[4.] Estude a série de potencia  $\mathlarger{\sum}_{n=0}^{\infty}\frac{(2x-3)^{n}}{\sqrt{n+1}}$.
\end{itemize}
\begin{itemize}
	\item[Opcional.] Enuncie o Teorema de Stokes.
\end{itemize}

	\flushbottom
	\flushright
     Êxitos...!!!
     \begin{center}
     	%\includegraphics[width=\linewidth/6]{logoUFPE.jpg}%LOGOTIPO DA INSTITUICAO
     	\vspace{50pt}
     	
     	\universidade
     	\par
     	\centro
     	\par
     	\departamento
     	\par
     	\curso
     	\par
     	\vspace{08pt}
     	\text{ Prova Final - Cálculo Diferencial e Integral 3}\\
     	\text{Prof. Fernando R. L. Contreras}	
     \end{center}
     
     \begin{flushleft}
     	Aluno(a):
     \end{flushleft}
 
 \begin{itemize}
 	\item[1.] Seja $\vec{E}(x,y,z)=\frac{q}{x^{2}+y^{2}+z^{2}}\frac{1}{\sqrt{x^{2}+y^{2}+z^{2}}}(x,y,z)$ o campo elétrico criado por uma carga $q$ localizada na origem. Calcule o fluxo de $\vec{E}$ através da superfície esférica de raio $r$ e centrada na origem, com normal $\vec{n}$ apontando para fora da esfera.
 \end{itemize}
 \begin{itemize}
 	\item[2.]  Seja $\vec{F}$ um campo de classe $C^{1}$ num aberto contendo a fronteira do cubo $0\leq x \leq 1$, $0\leq y \leq1$ e $0\leq z \leq1$. Seja $\vec{n}$ a normal apontando para fora do cubo. Mostre que $\mathlarger{\iint}_{S}rot(\vec{F})\cdot \vec{n}ds=0$ 
 \end{itemize}
 \begin{itemize}
 	\item [3.] Calcule $\mathlarger{\iint}_{S}rot(F)dS$, onde $S=\left\lbrace  (x,y,z)\in \mathbb{R}^{3}\quad \text{tal que}\quad x=-1+y^{2}+z^{2},\quad x\leq 0 \right\rbrace $ e o campo $\vec{F}$ é definido $\vec{F}(x,y,z)=(xz,ze^{x},-y)$.
 \end{itemize}
 \begin{itemize}
 	\item[4.] Estude a série de potencia  $\mathlarger{\sum}_{n=0}^{\infty}\frac{(2x-3)^{n}}{\sqrt{n+1}}$. 
 \end{itemize}
 \begin{itemize}
 	\item[Opcional.] Enuncie o Teorema de Stokes.
 \end{itemize}
\flushbottom
\flushright
Êxitos...!!!
\end{document}

