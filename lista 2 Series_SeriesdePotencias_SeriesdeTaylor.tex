
%%%%%%%%%%%%%%%%%%%%%%%%%%%%%%%%%%%%%%%%%%%%%%%%%%%%%%%%%%%%%%%%%%%%%%%%
%    Option test file, will be created during the first LaTeX run:
\begin{filecontents}{exercise.thm}
\def\th@exercise{%
  \normalfont % body font
  \thm@headpunct{:}%
}
\end{filecontents}
%%%%%%%%%%%%%%%%%%%%%%%%%%%%%%%%%%%%%%%%%%%%%%%%%%%%%%%%%%%%%%%%%%%%%%%%

\documentclass[12pt,openright,oneside,a4paper,english,french,spanish,brazil]{article}
% ---
% Pacotes básicos 
% ---
\usepackage{lmodern}			  % Usa a fonte Latin Modern			
\usepackage[T1]{fontenc}		  % Selecao de codigos de fonte.
\usepackage[utf8]{inputenc}	      % Codificacao do documento (conversão automática dos acentos)
\usepackage[top=20mm, bottom=20mm, left=20mm, right=20mm]{geometry}
\usepackage{lastpage}			  % Usado pela Ficha catalográfica
\usepackage{indentfirst}		  % Indenta o primeiro parágrafo de cada seção.
\usepackage{color}				  % Controle das cores
\usepackage{graphicx}			  % Inclusão de gráficos
\usepackage{microtype} 		      % para melhorias de justificação
\usepackage{booktabs}
\usepackage{multirow}
\usepackage[table]{xcolor}
\usepackage{subfig}
\usepackage{epstopdf}
\usepackage{hyperref}
\usepackage[mathcal]{eucal}
\usepackage{amsmath}               % great math stuff
\usepackage{amsfonts}              % for blackboard bold, etc
\usepackage{amsthm}                % better theorem environments
\usepackage{amssymb}
\usepackage{mathrsfs}
\DeclareMathAlphabet{\mathpzc}{OT1}{pzc}{m}{it}
\usepackage{undertilde}            % botar tilde embaixo da letra
\usepackage{mathptmx}          % fonte
\usepackage{latexsym}
\usepackage{makeidx}            % para definir o índice
\usepackage{epsfig}             % para introduzir figuras no formato eps
\usepackage{graphicx,color}     % permite a inclusao de figuras
\usepackage{verbatim}
\usepackage{gensymb}
\usepackage{titling}
\newcommand{\subtitle}[1]{%
	\posttitle{%
		\par\end{center}
	\begin{center}\Large#1\end{center}
	\vskip0.5em}%
}



\newtheorem{df}{Definição}
\newtheorem{ex}{Exemplo}
\newtheorem{teo}{Teorema}

\newtheoremstyle{note}% name
  {3pt}%      Space above
  {3pt}%      Space below
  {}%         Body font
  {}%         Indent amount (empty = no indent, \parindent = para indent)
  {\itshape}% Thm head font
  {:}%        Punctuation after thm head
  {.5em}%     Space after thm head: " " = normal interword space;
        %       \newline = linebreak
  {}%         Thm head spec (can be left empty, meaning `normal')

\theoremstyle{note}
\newtheorem{note}{Note}

\newtheoremstyle{citing}% name
  {3pt}%      Space above, empty = `usual value'
  {3pt}%      Space below
  {\itshape}% Body font
  {}%         Indent amount (empty = no indent, \parindent = para indent)
  {\bfseries}% Thm head font
  {.}%        Punctuation after thm head
  {.5em}%     Space after thm head: " " = normal interword space;
        %       \newline = linebreak
  {\thmnote{#3}}% Thm head spec

\theoremstyle{citing}
\newtheorem*{varthm}{}% all text supplied in the note

\newtheoremstyle{break}% name
  {9pt}%      Space above, empty = `usual value'
  {9pt}%      Space below
  {\itshape}% Body font
  {}%         Indent amount (empty = no indent, \parindent = para indent)
  {\bfseries}% Thm head font
  {.}%        Punctuation after thm head
  {\newline}% Space after thm head: \newline = linebreak
  {}%         Thm head spec

\theoremstyle{break}
\newtheorem{bthm}{B-Theorem}

\theoremstyle{exercise}
\newtheorem{exer}{Exercise}

\swapnumbers
\theoremstyle{plain}
\newtheorem{thmsw}{Theorem}[section]
%\newtheorem{corsw}[thm]{Corollary}
\newtheorem{propsw}{Proposition}
%\newtheorem{lemsw}[thm]{Lemma}

%    Because the amsmath pkg is not used, we need to define a couple of
%    commands in more primitive terms.
\let\lvert=|\let\rvert=|
\newcommand{\Ric}{\mathop{\mathrm{Ric}}\nolimits}

%    Dispel annoying problem of slightly overlong lines:
\addtolength{\textwidth}{8pt}

\title{ \textbf{Notas de Aula}}

\author{\textbf{Fernando Contreras}\\
	\large Nucleo de Tecnologia\\
	Universidade Federal de Pernambuco (UFPE)}



\begin{document}
	\begin{center}
		Universidade Federal de Pernambuco (UFPE)\\
		Centro Acadêmico do Agreste\\
		Núcleo de Tecnologia\\
		
		Lista 2 de Calculo Diferencial e Integral 3\\
		Prof. Fernando RL Contreras
	\end{center}


Sejam os seguintes problemas relativos a Séries, Séries de potências e Séries de Taylor.

%\begin{multicols}{2}

\begin{itemize}
	\item[1.] Determine se a série é convergente ou divergente 
	\begin{itemize}
	\item[a.] $\sum_{n=1}^{\infty}\frac{1}{2n}$
	%problema 21 pag. 658, stewart	
	\item[b.] $\sum_{n=1}^{\infty}\frac{1+3^{n}}{2^{n}}$
	%problema 26 pag. 658, stewart	
    \end{itemize}	
\end{itemize}
\begin{itemize}
	\item[2.] Pelo teste da integral determine se a série é convergente ou divergente.   
	\begin{itemize}
		\item[a.] $\sum_{n=1}^{\infty}\frac{1}{n^{2}-4n+5}$
		%problema 20 pag. 667, stewart	
		\item[b.] $\sum_{n=1}^{\infty}\frac{e^{1/2}}{n^{2}}$
		%problema 23 pag. 667, stewart	
	\end{itemize}
\end{itemize}
\begin{itemize}
	\item [3.] Pelo teste de comparação ou teste do limite determine a convergência ou divergência das séries: 
	\begin{itemize}
		\item[a.] $\sum_{n=1}^{\infty}\frac{n-1}{n^{2}\sqrt{n}}$
		%problema 6 pag. 672, stewart	
		\item[b.] $\sum_{n=1}^{\infty}\frac{1+4^{n}}{1+3^{n}}$
		%problema 19 pag. 672, stewart	
		\item[c.] $\sum_{n=1}^{\infty}\frac{n+5}{\sqrt[3]{n^{7}+n^{2}}}$
		%problema 26 pag. 672, stewart
	\end{itemize}
		
\end{itemize}
\begin{itemize}
	\item[4.] Teste a Série quanto a convergência ou divergência
	\begin{itemize}
	\item[a.] $\sum_{n=1}^{\infty}\frac{\cos (n\pi)}{n^{3/4}}$
	%problema 15 pag. 677, stewart	
	\item[b.] $\sum_{n=1}^{\infty}(-1)^{n}\sin (\frac{\pi}{n})$
	%problema 17 pag. 677, stewart	
    \end{itemize}
\end{itemize}
\begin{itemize}
	\item[5.] Os termos da série são definidos recursivamente pelas equações $a_{1}=2$, $a_{n+1}=\frac{5n+1}{4n-3}a_{n}$. Determine se $\sum_{}^{} a_{n}$ converge ou diverge.
	%problema 29 pag. 683, stewart
\end{itemize}

\begin{itemize}
	\item[6.] Uma série $\sum_{}^{} a_{n}$ é definida pelas equações $a_{1}=2$, $a_{n+1}=\frac{2+\cos (n)}{\sqrt{n}}a_{n}$. Determine se $\sum_{}^{} a_{n}$ converge ou diverge.
	% %problema 30 pag. 683, stewart
\end{itemize}
\begin{itemize}
	\item[7.] Encontre o raio de convergência e o intervalo de convergência das série de potência.
	\begin{itemize}
		\item[a.] $\sum_{n=2}^{\infty}\frac{(-1)^{n}x^{n}}{4^{n}\ln (n)}$
		%problema 13 pag. 691, stewart	
		\item[b.] $\sum_{n=0}^{\infty}\frac{(-1)^{n}(x-3)^{n}}{2n+1}$
		%problema 16 pag. 691, stewart	
		\item[c.] $\sum_{n=1}^{\infty}\frac{(x-2)^{n}}{n^{n}}$
		%problema 19 pag. 691, stewart
	    \item[d.] $\sum_{n=1}^{\infty}\frac{n^{2}x^{n}}{2.4.6. ... .(2n)}$
		%problema 24 pag. 691, stewart
	\end{itemize}
\end{itemize}
\begin{itemize}
	\item[8.]  Se $k$ for um inteiro positivo, encontre o raio de convergência  ou divergência da série $\sum_{n=0}^{\infty}\frac{(n!)^{k}}{(kn)!}x^{n}$ 
	%problema 31 pag. 691, stewart
\end{itemize}
\begin{itemize}
	\item[9.] A função $J_{1}$ definida por $J_{1}=\sum_{n=0}^{\infty}\frac{(-1)^{n}x^{2n+1}}{n!(n+1)!2^{2n+1}}$ denominado função de Bessel de ordem 1. Encontre seu domínio.
	%problema 35 pag. 692, stewart
\end{itemize}
\begin{itemize}
	\item[10.]Encontre uma representação em serie de potencia para a função e determine o intervalo de convergência.
	 \begin{itemize}
	 	\item[a.] $f(x)=\frac{1+x}{1-x}$
	 	%problema 9 pag. 697, stewart	
	 	\item[b.] $f(x)=\frac{1}{x+10}$
	 	%problema 6 pag. 697, stewart	
	 	\item[c.] $f(x)=\frac{x}{9+x^{2}}$
	 	%problema 7 pag. 697, stewart
	 	\item[d.] $f(x)=\frac{3}{1-x^{4}}$
	 	%problema 4 pag. 697, stewart
	 \end{itemize}
\end{itemize}
\begin{itemize}
	\item[11.] Use a derivação para encontrar a representação em série de potência para:
	\begin{itemize}
		\item[a.] $f(x)=\frac{1}{(1+x)^{2}}$
		%problema 13a pag. 697, stewart	
		\item[b.] $f(x)=\frac{1}{(1+x)^{3}}$
		%problema 13b pag. 697, stewart	
		\item[c.] $f(x)=\frac{x^{2}}{(1+x)^{3}}$
		%problema 13c pag. 697, stewart
	\end{itemize}
Qual é o raio de convergência?
\end{itemize}
\begin{itemize}
	\item [12.] Calcule a integral indefinida como uma série de potências. Qual é o raio de convergência?
	\begin{itemize}
		\item[a.] $\int\frac{t}{1-t^{8}}dt$
		%problema 23 pag. 697, stewart	
		\item[b.] $\int\frac{\ln (1-t)}{t}dt$
		%problema 24 pag. 697, stewart	
	\end{itemize} 
\end{itemize}
\begin{itemize}
	\item [13.] Mostre que a função $f(x)=\sum_{n=0}^{\infty}\frac{x^{n}}{n!}$ é solução da equação diferencial $f'(x)=f(x)$. E mostre que $f(x)=e^{x}$.
	%problema 35 pag. 698, stewart
\end{itemize}
\begin{itemize}
	\item [14.] Mostre que a função $f(x)=\sum_{n=0}^{\infty} \frac{(-1)^{n}x^{2n}}{(2n)!}$ é uma solução da equação diferencial $f''(x)+f(x)=0$
		%problema 32 pag. 697, stewart
\end{itemize}
\begin{itemize}
	\item [14.] Determine o polinômio de Taylor de ordem 1 da função dada, em volta do ponto $(x_{0},y_{0})$ dado por 
	\begin{itemize}
		\item[a.] $f(x,y)=e^{x+5y}$ e $(x_{0},y_{0})=(0,0)$
		%problema a pag. 302, Guidorizzi vol2
		\item[b.] $f(x,y)=x^{3}+y^{3}-x^{2}+4y$ e $(x_{0},y_{0})=(1,1)$
		%problema b pag. 302, Guidorizzi vol2
		\item[c.] $f(x,y)=\sin (3x+4y)$ e $(x_{0},y_{0})=(0,0)$
		%problema c pag. 302, Guidorizzi vol2	
	\end{itemize}    
\end{itemize}
\begin{itemize}
	\item [15.] Determine o polinômio de Taylor de ordem 2 da função dada, em volta do ponto $(x_{0},y_{0})$ dado por 
	\begin{itemize}
		\item[a.] $f(x,y)=x\sin (y)$ e $(x_{0},y_{0})=(0,0)$
		%problema a pag. 305, Guidorizzi vol2
		\item[b.] $f(x,y)=x^{3}+2x^{2}y+3y^{3}+x-y$ e $(x_{0},y_{0})=(1,1)$
		%problema b pag. 305, Guidorizzi vol2
	\end{itemize}
\end{itemize}
\begin{itemize}
	\item [16.] Seja $P_{2}(x,y)$ o polinômio de Taylor de ordem 2 de $f(x,y)=x\sin (y)$ em volta de $(0,0)$. Mostre que $\left|f(x,y)-P_{2}(x,y) \right|<\frac{\left| y\right| ^{2}}{2}\left[\left| x\right| +\frac{1}{3} \left|y \right| \right]  $
	%problema 3 pag. 305, Guidorizzi vol2
\end{itemize}
\end{document}
