\documentclass[oneside,a4paper,12pt]{article}
\usepackage[english,brazilian]{babel}
\usepackage[alf]{abntex2cite}
\usepackage[utf8]{inputenc}
\usepackage[T1]{fontenc}

\usepackage{lastpage}			  % Usado pela Ficha catalográfica
\usepackage{indentfirst}		  % Indenta o primeiro parágrafo de cada
\usepackage[top=20mm, bottom=20mm, left=20mm, right=20mm]{geometry}
\usepackage{framed}
\usepackage{booktabs}

\usepackage{float}
\usepackage{color}				  % Controle das cores
\usepackage{graphicx}			  % Inclusão de gráficos
\usepackage{microtype} 		      % para melhorias de justificação
\usepackage{booktabs}
\usepackage{multirow}
\usepackage[table]{xcolor}
\usepackage{subfig}
\usepackage{epstopdf}
\usepackage{hyperref}

\usepackage[mathcal]{eucal}
\usepackage{amsmath}               % great math stuff
\usepackage{amsfonts}              % for blackboard bold, etc
\usepackage{amsthm}                % better theorem environments
\usepackage{amssymb}
\usepackage{mathrsfs}
\DeclareMathAlphabet{\mathpzc}{OT1}{pzc}{m}{it}
\usepackage{undertilde}            % botar tilde embaixo da letra
\usepackage{mathptmx}          % fonte
\usepackage{graphicx}
\graphicspath{{./Figuras/}}    
\definecolor{shadecolor}{rgb}{0.8,0.8,0.8}


%FAZ EDICOES AQUI (somente no conteudo que esta entre entre as ultimas  chaves de cada linha!!!)
\newcommand{\universidade}{Universidade Federal de Pernambuco}
\newcommand{\centro}{Centro Acadêmico do Agreste}
\newcommand{\departamento}{Núcleo de Tecnologia}
\newcommand{\curso}{Engenharia de Produção}
\newcommand{\professor}{Fernando R. L. Contreras}
\newcommand{\disciplina}{Cálculo Diferencial e Integral 3}
%ATE AQUI !!!

\begin{document}
	\pagestyle{empty}
	
	\begin{center}
	%\includegraphics[width=\linewidth/6]{logoUFPE.jpg}%LOGOTIPO DA INSTITUICAO
	 	\vspace{0pt}
	 	
		\universidade
		\par
		\centro
		\par
		\departamento
		\par
		\curso
		\par
		\vspace{08pt}
		\text{ Prova 3 - Cálculo Diferencial e Integral 3}\\
		\text{Prof. Fernando R. L. Contreras}	
	\end{center}
	
	%\vspace{0.5pt}
	
	\begin{flushleft}
		\textbf{Aluno(a):}
	\end{flushleft}
	
\begin{itemize}
\item[1.]  Se $\textbf{r}=(x,y,z)$ e $\textbf{a}\times \textbf{r}=(P,Q,R)$, sendo $\textbf{a}$ um vetor constante, mostrar que $\int_{C} P\mathrm{d}x+Q\mathrm{d}y+R\mathrm{d}z=2\iint_{S}\textbf{a}\cdot \textbf{n}\mathrm{d}s$, sendo $C$ a curva que limita uma superfície paramétrica $S$ e $\textbf{n}$ a normal unitária a $S$ conveniente.\textit{Sug. Utilize o Teorema de Stokes}.
\end{itemize}
\begin{itemize}
\item[2.] Seja o campo vetorial $F=(-2x+\sin(z^{3}),\quad 4y,\quad x^{3}+6z)$ e a superfície aberta $S_{1}$: $y^{2}=x^{2}+z^{2}$, $(0\leq y \leq 3)$ orientada de modo que o vetor normal exterior tenha segunda componente negativa. Calcule o fluxo de $F$ através dela. 
 \end{itemize}
 \begin{itemize}
\item [3.] Dados dois campos escaleres $u$ e $v$, continuamente diferenciáveis num conjunto aberto que contém o disco circular $R$ cuja fronteira é a circunferência $x^{2}+y^{2}=1$, definem-se dois campos vetoriais $\textbf{f}$ e $\textbf{g}$ do mogo seguinte: $\textbf{f}(x,y)=(v(x,y), u(x,y))$, $\textbf{g}(x,y)=(\frac{\partial u}{\partial x}-\frac{\partial u}{\partial y}, \frac{\partial v}{\partial x}-\frac{\partial v}{\partial y})$. Determine o valor do integral duplo $\iint_{R} \textbf{f}\cdot\textbf{g}\mathrm{d}x\mathrm{d}y$ se é sabido que sobre a fronteira de $R$ se tem $u(x,y)=1$ e $v(x,y)=y$. \textit{Sug. Utilize o Teorema de Green}.
\end{itemize}
\begin{itemize}
\item[4.] Calcule a integral $\int_{\Gamma}2(x+y^{2}) \mathrm{d}x+(4xy+\cos y)\mathrm{d}y$, onde $\Gamma$ é uma curva arbitraria suave por partes que une os pontos $(1,0)$ e $(\xi, \eta)$.
\end{itemize}
\begin{itemize}
	\item[Opcional.] Enuncie e mostre o Teorema de Fundamental de Integrais de Linha.
\end{itemize}

	\flushbottom
	\flushright
     \begin{center}
     	%\includegraphics[width=\linewidth/6]{logoUFPE.jpg}%LOGOTIPO DA INSTITUICAO
     	\vspace{10pt}
     	
     	\universidade
     	\par
     	\centro
     	\par
     	\departamento
     	\par
     	\curso
     	\par
     	\vspace{08pt}
     	\text{ Prova 3 - Cálculo Diferencial e Integral 3}\\
     	\text{Prof. Fernando R. L. Contreras}	
     \end{center}
     
     \begin{flushleft}
     	\textbf{Aluno(a):}
     \end{flushleft}
 
 \begin{itemize}
 	\item[1.] Se $\textbf{r}=(x,y,z)$ e $\textbf{a}\times \textbf{r}=(P,Q,R)$, sendo $\textbf{a}$ um vetor constante, mostrar que $\int_{C} P\mathrm{d}x+Q\mathrm{d}y+R\mathrm{d}z=2\iint_{S}\textbf{a}\cdot \textbf{n}\mathrm{d}s$, sendo $C$ a curva que limita uma superfície paramétrica $S$ e $\textbf{n}$ a normal unitária a $S$ conveniente.\textit{Sug. Utilize o Teorema de Stokes}.
 \end{itemize} 
 \begin{itemize}
 	\item[2.] Seja o campo vetorial $F=(-2x+\sin(z^{3}),\quad 4y,\quad x^{3}+6z)$ e a superfície aberta $S_{1}$: $y^{2}=x^{2}+z^{2}$, $(0\leq y \leq 3)$ orientada de modo que o vetor normal exterior tenha segunda componente negativa. Calcule o fluxo de $F$ através dela. 
 \end{itemize}
 \begin{itemize}
 	\item [3.] Dados dois campos escaleres $u$ e $v$, continuamente diferenciáveis num conjunto aberto que contém o disco circular $R$ cuja fronteira é a circunferência $x^{2}+y^{2}=1$, definem-se dois campos vetoriais $\textbf{f}$ e $\textbf{g}$ do mogo seguinte: $\textbf{f}(x,y)=(v(x,y), u(x,y))$, $\textbf{g}(x,y)=(\frac{\partial u}{\partial x}-\frac{\partial u}{\partial y}, \frac{\partial v}{\partial x}-\frac{\partial v}{\partial y})$. Determine o valor do integral duplo $\iint_{R} \textbf{f}\cdot\textbf{g}\mathrm{d}x\mathrm{d}y$ se é sabido que sobre a fronteira de $R$ se tem $u(x,y)=1$ e $v(x,y)=y$. \textit{Sug. Utilize o Teorema de Green}.
 \end{itemize}
 \begin{itemize}
 	\item[4.]Calcule a integral $\int_{\Gamma}2(x+y^{2}) \mathrm{d}x+(4xy+\cos y)\mathrm{d}y$, onde $\Gamma$ é uma curva arbitraria suave por partes que une os pontos $(1,0)$ e $(\xi, \eta)$.
 \end{itemize}
 \begin{itemize}
 	\item[Opcional.] Enuncie e mostre o Teorema de Fundamental de Integrais de Linha.
 \end{itemize}
\flushbottom
\flushright
\end{document}

