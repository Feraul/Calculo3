\documentclass[oneside,a4paper,12pt]{article}
\usepackage[english,brazilian]{babel}
\usepackage[alf]{abntex2cite}
\usepackage[utf8]{inputenc}
\usepackage[T1]{fontenc}

\usepackage{lastpage}			  % Usado pela Ficha catalográfica
\usepackage{indentfirst}		  % Indenta o primeiro parágrafo de cada
\usepackage[top=20mm, bottom=20mm, left=20mm, right=20mm]{geometry}
\usepackage{framed}
\usepackage{booktabs}

\usepackage{float}
\usepackage{color}				  % Controle das cores
\usepackage{graphicx}			  % Inclusão de gráficos
\usepackage{microtype} 		      % para melhorias de justificação
\usepackage{booktabs}
\usepackage{multirow}
\usepackage[table]{xcolor}
\usepackage{subfig}
\usepackage{epstopdf}
\usepackage{hyperref}

\usepackage[mathcal]{eucal}
\usepackage{amsmath}               % great math stuff
\usepackage{amsfonts}              % for blackboard bold, etc
\usepackage{amsthm}                % better theorem environments
\usepackage{amssymb}
\usepackage{mathrsfs}
\DeclareMathAlphabet{\mathpzc}{OT1}{pzc}{m}{it}
\usepackage{undertilde}            % botar tilde embaixo da letra
\usepackage{mathptmx}          % fonte
\usepackage{graphicx}
\graphicspath{{./Figuras/}}    
\definecolor{shadecolor}{rgb}{0.8,0.8,0.8}


%FAZ EDICOES AQUI (somente no conteudo que esta entre entre as ultimas  chaves de cada linha!!!)
\newcommand{\universidade}{Universidade Federal de Pernambuco}
\newcommand{\centro}{Centro Acadêmico do Agreste}
\newcommand{\departamento}{Núcleo de Tecnologia}
\newcommand{\curso}{Engenharia de Produção}
\newcommand{\professor}{Fernando R. L. Contreras}
\newcommand{\disciplina}{Cálculo Diferencial e Integral 3}
%ATE AQUI !!!

\begin{document}
	\pagestyle{empty}
	
	\begin{center}
	%\includegraphics[width=\linewidth/6]{logoUFPE.jpg}%LOGOTIPO DA INSTITUICAO
	 	\vspace{0pt}
	 	
		\universidade
		\par
		\centro
		\par
		\departamento
		\par
		\curso
		\par
		\vspace{08pt}
		\text{ Prova 2 - Cálculo Diferencial e Integral 3}\\
		\text{Prof. Fernando R. L. Contreras}	
	\end{center}
	
	%\vspace{0.5pt}
	
	\begin{flushleft}
		\textbf{Aluno(a):}
	\end{flushleft}
	
\begin{itemize}
\item[1.] Uma determinada empresa produz dois produtos cujas quantidades são indicadas por $x$ e $y$. Tais produtos são oferecidos ao mercado consumidor a preços unitários $p_{1}=120-2x$ e $p_{2}=200-y$. O custo total da empresa para produzir e vender quantidades $x$ e $y$ dos produtos é dado por $C=x^{2}+2y^{2}+2xy$. Admitindo que toda produção da empresa seja absorvida pelo mercado, determine a produção que maximiza o lucro. Qual o lucro máximo?
\end{itemize}
\begin{itemize}
\item[2.] Calcule $\iint_{E} x^{2}dV$ sendo $E$ o sólido limitado pelos planos. $x+y+z=4$; $x=1$; $y=3$; $z=3$ no primeiro octante e os planos coordenados. Esboce o gráfico. 
 \end{itemize}
 \begin{itemize}
\item [3.] Calcule o volume do sólido $S$ limitado inferiormente pela equação do cone $z^{2}=x^{2}+y^{2}$, e interior da esfera $x^{2}+y^{2}+z^{2}=2az$, $a>0$. Esboce o gráfico.
\end{itemize}
\begin{itemize}
\item[4.] Um fabricante de embalagens deve fabricar um lote de caixas retangulares de volume $V=64cm^{3}$. Se o custo do material usado na fabricação da caixa é de $ 0,5$ Reais por centímetro quadrado, determinar as dimensões da caixa que tornem mínimo o custo do material usado em sua fabricação .
\end{itemize}
\begin{itemize}
	\item[Opcional.] Generalize o método de multiplicadores de Lagrange para uma função com $n$ variáveis.
\end{itemize}

	\flushbottom
	\flushright
     \begin{center}
     	%\includegraphics[width=\linewidth/6]{logoUFPE.jpg}%LOGOTIPO DA INSTITUICAO
     	\vspace{10pt}
     	
     	\universidade
     	\par
     	\centro
     	\par
     	\departamento
     	\par
     	\curso
     	\par
     	\vspace{08pt}
     	\text{ Prova 2 - Cálculo Diferencial e Integral 3}\\
     	\text{Prof. Fernando R. L. Contreras}	
     \end{center}
     
     \begin{flushleft}
     	\textbf{Aluno(a):}
     \end{flushleft}
 
 \begin{itemize}
 	\item[1.] Uma determinada empresa produz dois produtos cujas quantidades são indicadas por $x$ e $y$. Tais produtos são oferecidos ao mercado consumidor a preços unitários $p_{1}=120-2x$ e $p_{2}=200-y$. O custo total da empresa para produzir e vender quantidades $x$ e $y$ dos produtos é dado por $C=x^{2}+2y^{2}+2xy$. Admitindo que toda produção da empresa seja absorvida pelo mercado, determine a produção que maximiza o lucro. Qual o lucro máximo?.
 \end{itemize}
 \begin{itemize}
 	\item[2.] Calcule $\iint_{E} x^{2}dV$ sendo $E$ o sólido limitado pelos planos. $x+y+z=4$; $x=1$; $y=3$; $z=3$ no primeiro octante e os planos coordenados. Esboce o gráfico.. 
 \end{itemize}
 \begin{itemize}
 	\item [3.] Calcule o volume do sólido $S$ limitado inferiormente pela equação do cone $z^{2}=x^{2}+y^{2}$, e interior da esfera $x^{2}+y^{2}+z^{2}=2az$, $a>0$. Esboce o gráfico.
 \end{itemize}
 \begin{itemize}
 	\item[4.] Um fabricante de embalagens deve fabricar um lote de caixas retangulares de volume $V=64cm^{3}$. Se o custo do material usado na fabricação da caixa é de $ 0,5$ Reais por centímetro quadrado, determinar as dimensões da caixa que tornem mínimo o custo do material usado em sua fabricação.
 \end{itemize}
 \begin{itemize}
 	\item[Opcional.] Generalize o método de multiplicadores de Lagrange para uma função com $n$ variáveis.
 \end{itemize}
\flushbottom
\flushright
\end{document}

