\documentclass[oneside,a4paper,12pt]{article}
\usepackage[english,brazilian]{babel}
\usepackage[alf]{abntex2cite}
\usepackage[utf8]{inputenc}
\usepackage[T1]{fontenc}

\usepackage{lastpage}			  % Usado pela Ficha catalográfica
\usepackage{indentfirst}		  % Indenta o primeiro parágrafo de cada
\usepackage[top=20mm, bottom=20mm, left=20mm, right=20mm]{geometry}
\usepackage{framed}
\usepackage{booktabs}

\usepackage{float}
\usepackage{color}				  % Controle das cores
\usepackage{graphicx}			  % Inclusão de gráficos
\usepackage{microtype} 		      % para melhorias de justificação
\usepackage{booktabs}
\usepackage{multirow}
\usepackage[table]{xcolor}
\usepackage{subfig}
\usepackage{epstopdf}
\usepackage{hyperref}

\usepackage[mathcal]{eucal}
\usepackage{amsmath}               % great math stuff
\usepackage{amsfonts}              % for blackboard bold, etc
\usepackage{amsthm}                % better theorem environments
\usepackage{amssymb}
\usepackage{mathrsfs}
\DeclareMathAlphabet{\mathpzc}{OT1}{pzc}{m}{it}
\usepackage{undertilde}            % botar tilde embaixo da letra
\usepackage{mathptmx}          % fonte
\usepackage{graphicx}
\graphicspath{{./Figuras/}}    
\definecolor{shadecolor}{rgb}{0.8,0.8,0.8}


%FAZ EDICOES AQUI (somente no conteudo que esta entre entre as ultimas  chaves de cada linha!!!)
\newcommand{\universidade}{Universidade Federal de Pernambuco}
\newcommand{\centro}{Centro Acadêmico do Agreste}
\newcommand{\departamento}{Núcleo de Tecnologia}
\newcommand{\curso}{Engenharia Civil}
\newcommand{\professor}{Fernando R. L. Contreras}
\newcommand{\disciplina}{Cálculo Diferencial e Integral 3}
%ATE AQUI !!!

\begin{document}
	\pagestyle{empty}
	
	\begin{center}
	\includegraphics[width=\linewidth/6]{logoUFPE.jpg}%LOGOTIPO DA INSTITUICAO
	 	\vspace{0pt}
	 	
		\universidade
		\par
		\centro
		\par
		\departamento
		\par
		\curso
		\par
		\vspace{08pt}
		\large \textbf{Primeira Prova}
		
	\end{center}
	
	\vspace{1pt}
	
	\begin{tabular}{ |l|p{12cm}| }
		
		\hline
		\multicolumn{2}{|c|}{\textbf{Dados de Identificação}} \\
			\hline
		Disciplina:        &    \disciplina          \\
		\hline
		Professor:         &    \professor           \\
	\hline
	Aluno(a):         &\\
	
		\hline
	\end{tabular}
	
	\vspace{10pt}
	\textbf{Justifique todas as suas respostas. Você também será avaliado pela clareza e pela precisão da linguagem utilizada.}
\begin{itemize}
\item[1.] Estude a:
	\begin{itemize}
	\item[(a)]\textbf{(1.0)} Convergência, convergência absoluta ou divergência da série $\mathlarger{\sum}_{n=2}^{\infty}\frac{(-1)^{n}n^{3}}{n^{4}-1}$.
	\item[(b)]\textbf{(1.0)} Convergência ou divergência da série $\mathlarger{\sum}_{n=1}^{\infty} n^{2}(1-\cos(\frac{1}{n^{2}})) $.
	\end{itemize}
\end{itemize}
\begin{itemize}
\item[2.] Seja $f(x)=\mathlarger{\sum}_{n=1}^{\infty}\frac{(n+1)^{2}}{n^{3}}(x-2)^{n}$.
	\begin{itemize} 
	\item[(a)]\textbf{(1.5)} Determine os valores de $x\in\mathbb{R}$ tais que $f(x)$ é convergente.
	\item[(b)]\textbf{(0.5)} Determine $f^{(17)}(2)$, onde $f^{(17)}(x)$ é a décima sétima derivada de $f(x)$.
    \end{itemize}
 \end{itemize}
 \begin{itemize}
\item [3.] Mostre e Calcule:
	\begin{itemize}
	\item[(a)]\textbf{(1.0)} Seja $\left\lbrace a_{n}\right\rbrace $ tal que $a_{n+1}=\frac{3(1+a_{n})}{3+a_{n}}$, $a_{1}=3$. Mostre que $\left\lbrace a_{n}\right\rbrace $ tende a $\quad\sqrt[]{3}$.
	\item[(b)]\textbf{(1.0)} Calcule $\lim\limits_{n\longrightarrow \infty} n(a^{1/n}-1)$, $a>0$.
    \end{itemize}
\end{itemize}
\begin{itemize}
\item[4.]\textbf{(2.0)} Seja $T(x,y,z)=4+4x+4y+2z-x^{2}-2y^{2}-z^{2}$ a temperatura no elipsoide sólido $D={(x,y,z)\in\mathbb{R}^{3}: x^{2}+2y^{2}+z^{2}\leq 16}$.
\begin{itemize} 
	\item[(a)] Determine os pontos críticos da função $T(x,y,z)$ no interior de $D$.
	\item[(b)] Utilize o método de multiplicadores de Lagrange para determinar os pontos de temperatura máxima e mínima e o valor da temperatura máxima e mínima na fronteira de $D$ (isto é, na superfície do elipsoide sólido).
	\item[(c)] Utilize os resultados dos itens (a) e (b) para determinar os pontos de temperatura máxima e mínima e o valor da temperatura máxima e mínima em $D$.
\end{itemize}
\end{itemize}
\begin{itemize} 
	\item[5.] \textbf{(2.0)} Seja $P_{1}(x,y)$ o polinômio de ordem 1 de $f(x,y)=e^{x+5y}$ em volta de $(0,0)$. Mostre que $\left| f(x,y)-P_{1}(x,y) \right|<\frac{3}{2}\left(x+5y\right)^{2} $ para todo $(x,y)$, com $x+5y<1$.
\end{itemize} 
	\flushbottom
	\flushright
     Êxitos...!!!
\end{document}
