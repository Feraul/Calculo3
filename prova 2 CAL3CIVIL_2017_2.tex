\documentclass[oneside,a4paper,12pt]{article}
\usepackage[english,brazilian]{babel}
\usepackage[alf]{abntex2cite}
\usepackage[utf8]{inputenc}
\usepackage[T1]{fontenc}

\usepackage{lastpage}			  % Usado pela Ficha catalográfica
\usepackage{indentfirst}		  % Indenta o primeiro parágrafo de cada
\usepackage[top=20mm, bottom=20mm, left=20mm, right=20mm]{geometry}
\usepackage{framed}
\usepackage{booktabs}

\usepackage{float}
\usepackage{color}				  % Controle das cores
\usepackage{graphicx}			  % Inclusão de gráficos
\usepackage{microtype} 		      % para melhorias de justificação
\usepackage{booktabs}
\usepackage{multirow}
\usepackage[table]{xcolor}
\usepackage{subfig}
\usepackage{epstopdf}
\usepackage{hyperref}

\usepackage[mathcal]{eucal}
\usepackage{amsmath}               % great math stuff
\usepackage{amsfonts}              % for blackboard bold, etc
\usepackage{amsthm}                % better theorem environments
\usepackage{amssymb}
\usepackage{mathrsfs}
\DeclareMathAlphabet{\mathpzc}{OT1}{pzc}{m}{it}
\usepackage{undertilde}            % botar tilde embaixo da letra
\usepackage{mathptmx}          % fonte
\usepackage{graphicx}
\graphicspath{{./Figuras/}}    
\definecolor{shadecolor}{rgb}{0.8,0.8,0.8}


%FAZ EDICOES AQUI (somente no conteudo que esta entre entre as ultimas  chaves de cada linha!!!)
\newcommand{\universidade}{Universidade Federal de Pernambuco}
\newcommand{\centro}{Centro Acadêmico do Agreste}
\newcommand{\departamento}{Núcleo de Tecnologia}
\newcommand{\curso}{Engenharia Civil}
\newcommand{\professor}{Fernando R. L. Contreras}
\newcommand{\disciplina}{Cálculo Diferencial e Integral 3}
%ATE AQUI !!!

\begin{document}
	\pagestyle{empty}
	
	\begin{center}
	%\includegraphics[width=\linewidth/6]{logoUFPE.jpg}%LOGOTIPO DA INSTITUICAO
	 	\vspace{0pt}
	 	
		\universidade
		\par
		\centro
		\par
		\departamento
		\par
		\curso
		\par
		\vspace{08pt}
		\text{ Prova 2 - Cálculo Diferencial e Integral 3}\\
		\text{Prof. Fernando R. L. Contreras}	
	\end{center}
	
	%\vspace{0.5pt}
	
	\begin{flushleft}
		Aluno(a):
	\end{flushleft}
	
\begin{itemize}
\item[1.] Calcular $\mathlarger{\iint}_{S}F\cdot dS$, onde $F(x,y,z)=(z^{2}x,\frac{1}{3}y^{3}+\tan z, x^{2}z+y^{2})$ e $S$ é a metade de cima da esfera $x^{2}+y^{2}+z^{2}=1$.
\end{itemize}
\begin{itemize}
\item[2.] Se $S$ é uma esfera e satisfaz as hipóteses do Teorema de Stokes, verifique que  $\mathlarger{\iint}_{S}rot (F)\cdot dS=0$. 
 \end{itemize}
 \begin{itemize}
\item [3.] Verifique o teorema de Green, quando $P(x,y)=y^{2}e^{x}$, $Q(x,y)=x^{2}e^{y}$, $C$ consiste no segmento de reta de $(-1,1)$ a $(1,1)$ seguido pelo arco parabólico $y=2-x^{2}$ de $(1,1)$ a $(-1,1)$.
\end{itemize}
\begin{itemize}
\item[4.] Estude com relação a máximos e mínimos a função $f(x,y)=x^{2}-2xy+3y^{2}$ dada com a restrição $x^{2}+2y^{2}=1$.
\end{itemize}

	\flushbottom
	\flushright
     Êxitos...!!!
     \begin{center}
     	%\includegraphics[width=\linewidth/6]{logoUFPE.jpg}%LOGOTIPO DA INSTITUICAO
     	\vspace{50pt}
     	
     	\universidade
     	\par
     	\centro
     	\par
     	\departamento
     	\par
     	\curso
     	\par
     	\vspace{08pt}
     	\text{ Prova 2 - Cálculo Diferencial e Integral 3}\\
     	\text{Prof. Fernando R. L. Contreras}	
     \end{center}
     
     \begin{flushleft}
     	Aluno(a):
     \end{flushleft}
 
 \begin{itemize}
 	\item[1.] Calcular $\mathlarger{\iint}_{S}F\cdot dS$, onde $F(x,y,z)=(z^{2}x,\frac{1}{3}y^{3}+\tan z, x^{2}z+y^{2})$ e $S$ é a metade de cima da esfera $x^{2}+y^{2}+z^{2}=1$.
 \end{itemize}
 \begin{itemize}
 	\item[2.] Se $S$ é uma esfera e satisfaz as hipóteses do Teorema de Stokes, verifique que  $\mathlarger{\iint}_{S}rot (F)\cdot dS=0$. 
 \end{itemize}
 \begin{itemize}
 	\item [3.] Verifique o teorema de Green, quando $P(x,y)=y^{2}e^{x}$, $Q(x,y)=x^{2}e^{y}$, $C$ consiste no segmento de reta de $(-1,1)$ a $(1,1)$ seguido pelo arco parabólico $y=2-x^{2}$ de $(1,1)$ a $(-1,1)$.
 \end{itemize}
 \begin{itemize}
 	\item[4.] Estude com relação a máximos e mínimos a função $f(x,y)=x^{2}-2xy+3y^{2}$ dada com a restrição $x^{2}+2y^{2}=1$.
 \end{itemize}
 
\flushbottom
\flushright
Êxitos...!!!
\end{document}

