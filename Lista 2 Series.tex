\documentclass[oneside,a4paper,12pt]{article}
\usepackage[english,brazilian]{babel}
\usepackage[alf]{abntex2cite}
\usepackage[utf8]{inputenc}
\usepackage[T1]{fontenc}

\usepackage{lastpage}			  % Usado pela Ficha catalográfica
\usepackage{indentfirst}		  % Indenta o primeiro parágrafo de cada
\usepackage[top=5mm, bottom=5mm, left=10mm, right=10mm]{geometry}
\usepackage{framed}
\usepackage{booktabs}

\usepackage{float}
\usepackage{color}				  % Controle das cores
\usepackage{graphicx}			  % Inclusão de gráficos
\usepackage{microtype} 		      % para melhorias de justificação
\usepackage{booktabs}
\usepackage{multirow}
\usepackage[table]{xcolor}
\usepackage{subfig}
\usepackage{epstopdf}
\usepackage{hyperref}

\usepackage[mathcal]{eucal}
\usepackage{amsmath}               % great math stuff
\usepackage{amsfonts}              % for blackboard bold, etc
\usepackage{amsthm}                % better theorem environments
\usepackage{amssymb}
\usepackage{mathrsfs}
\DeclareMathAlphabet{\mathpzc}{OT1}{pzc}{m}{it}
\usepackage{undertilde}            % botar tilde embaixo da letra
\usepackage{mathptmx}          % fonte
\usepackage{graphicx}
\usepackage{multicol}
\graphicspath{{./Figuras/}}    
\definecolor{shadecolor}{rgb}{0.8,0.8,0.8}


%%FAZ EDICOES AQUI (somente no conteudo que esta entre entre as ultimas  chaves de cada linha!!!)
\newcommand{\universidade}{Universidade Federal de Pernambuco}
\newcommand{\centro}{Centro Acadêmico do Agreste}
\newcommand{\departamento}{Núcleo de Tecnologia}
%\newcommand{\curso}{Engenharia de Produção}
\newcommand{\professor}{Fernando R. L. Contreras}
\newcommand{\disciplina}{Cálculo Diferencial e Integral 3}
%%ATE AQUI !!!

\begin{document}
	\tiny
	\pagestyle{empty}
	
	\begin{center}
	%\includegraphics[width=\linewidth/6]{logoUFPE.jpg}%LOGOTIPO DA INSTITUICAO
	 	\vspace{0pt}
	 	
		\universidade
		\par
		\centro
		\par
		\departamento
%		\par
%		\curso
		\par
		\vspace{08pt}
		\text{Cálculo Diferencial e Integral 3}\\
		\text{Prof. Fernando R. L. Contreras}\\
			\text{Lista 2 - Séries}\\	
	\end{center}
	
	%\vspace{0.5pt}
	
%	\begin{flushleft}
%		Aluno(a):
%	\end{flushleft}
Nos problemas de 1 a 29, determine se a série infinita dada converge ou diverge. Se converge, determine sua soma

\begin{multicols}{2}
	
\begin{itemize}
\item[1.] $1+\frac{1}{3}+\frac{1}{9} +...+\frac{1}{3^{n}}+...$.
\end{itemize}
\begin{itemize}
\item[2.] $1+e^{-1}+e^{-2}+ ...+e^{-n}+...$. 
 \end{itemize}
 \begin{itemize}
\item [3.]$1+3+5+7+...+(2n-1)+...$.
\end{itemize}
\begin{itemize}
\item[4.] $\frac{1}{2}+\frac{1}{\sqrt{2}}+\frac{1}{\sqrt[3]{2}}+...+\frac{1}{\sqrt[n]{2}}+...$.
\end{itemize}
\begin{itemize}
\item[5.] $1-2+4-8+...+(-2)^{n}+...$.
\end{itemize}
\begin{itemize}
	\item[6.] $1-\frac{1}{4}+\frac{1}{16}-...+(-\frac{1}{4})^{n}+...$.
\end{itemize}
\begin{itemize}
	\item[7.] $4+\frac{4}{3}+\frac{4}{9}+\frac{4}{27}+...+\frac{4}{3^{n}}+...$.
\end{itemize}
\begin{itemize}
	\item[8.] $\frac{1}{3}+\frac{2}{9}+\frac{4}{27}+...+\frac{2^{n-1}}{3^{n}}+...$.
\end{itemize}
\begin{itemize}
	\item[9.] $1+(1.01)+(1.01)^{2}+(1.01)^{3}+...+(1.01)^{n}+...$.
\end{itemize}
\begin{itemize}
	\item[10.] $1+\frac{1}{\sqrt{2}}+\frac{1}{\sqrt[3]{3}}+...+\frac{1}{\sqrt[n]{n}}+...$.
\end{itemize}
\begin{itemize}
	\item[11.] $\sum_{n=0}^{\infty}\frac{(-1)^{n}n}{n+1}$.
\end{itemize}
\begin{itemize}
	\item [12.] $\sum_{n=1}^{\infty}(\frac{e}{10})^{n}$
\end{itemize}
\begin{itemize}
	\item [13.] $\sum_{n=0}^{\infty}(-1)^{n}(\frac{3}{e})^{n}$
\end{itemize}
\begin{itemize}
	\item [14.] $\sum_{n=0}^{\infty}\frac{3^{n}-2^{n}}{4^{n}}$
\end{itemize}
\begin{itemize}
	\item [15.] $\sum_{n=1}^{\infty}(\sqrt{2})^{1-n}$
\end{itemize}
\begin{itemize}
	\item [16.] $\sum_{n=1}^{\infty}(\frac{1}{2}-\frac{1}{2^{n}})$
\end{itemize}
\begin{itemize}
	\item [17.] $\sum_{n=1}^{\infty}\frac{n}{10n+17}$
\end{itemize}
\begin{itemize}	
	\item [18.] $\sum_{n=1}^{\infty}\frac{\sqrt{n}}{\ln (n+1)}$
\end{itemize}
\begin{itemize}
	\item [19.] $\sum_{n=1}^{\infty}(5^{-n}-7^{-n})$
\end{itemize}
\begin{itemize}
	\item [20.] $\sum_{n=1}^{\infty}(\frac{e}{\pi})^{n}$
\end{itemize}
\begin{itemize}
	\item [21.] $\sum_{n=1}^{\infty}(\frac{\pi}{e})^{n} $
\end{itemize}
\begin{itemize}
	\item [22.] $\sum_{n=0}^{\infty}(\frac{100}{99})^{n}$
\end{itemize}
\begin{itemize}
	\item [23.] $\sum_{n=0}^{\infty}(\frac{99}{100})^{n}$
\end{itemize}	
\begin{itemize}
	\item [24.] $\sum_{n=0}^{\infty}\frac{1+2^{n}+3^{n}}{5^{n}}$
\end{itemize}
\begin{itemize}
	\item [25.] $\sum_{n=0}^{\infty} \frac{1+2^{n}+5^{n}}{3^{n}}$
\end{itemize}
\begin{itemize}
	\item [26.] $\sum_{n=0}^{\infty}\frac{7\times 5^{n}+3\times11^{n}}{13^{n}}$
\end{itemize}
\begin{itemize}
	\item [27.] $\sum_{n=1}^{\infty}\sqrt[n]{2}$
\end{itemize}
\begin{itemize}
	\item [28.] $\sum_{n=1}^{\infty} \left[ (\frac{7}{11})^{n}-(\frac{3}{5})^{n}\right] $
\end{itemize}
\begin{itemize}
	\item [29.] $\sum_{n=0}^{\infty}\frac{1}{1+(\frac{1}{10})^{n}}$
\end{itemize}
\begin{itemize}
	\item [25.] Mostre que: se $\sum a_{n}$ diverge e $c$ é uma constante distinto de zero, então $\sum ca_{n}$ diverge.
\end{itemize}
\begin{itemize}
	\item [26.] Suponha que $\sum a_{n}$ converge e que $\sum b_{n}$. Mostre que $\sum (a_{n}+b_{n})$ diverge.
\end{itemize}
\begin{itemize}
	\item [27.] Sejam $S_{n}$ e $T_{n}$ a n-ésima soma parcial de $\sum a_{n}$ e $\sum b_{n}$ respectivamente. Suponha que $a_{n}=b_{n}$ para todo $n>k$. Mostre que $S_{n}-T_{n}=S_{k}-T_{k}$ se $n>k$. 
\end{itemize}
\end{multicols}    
\end{document}

