\documentclass[oneside,a4paper,12pt]{article}
\usepackage[english,brazilian]{babel}
\usepackage[alf]{abntex2cite}
\usepackage[utf8]{inputenc}
\usepackage[T1]{fontenc}

\usepackage{lastpage}			  % Usado pela Ficha catalográfica
\usepackage{indentfirst}		  % Indenta o primeiro parágrafo de cada
\usepackage[top=20mm, bottom=20mm, left=20mm, right=20mm]{geometry}
\usepackage{framed}
\usepackage{booktabs}

\usepackage{float}
\usepackage{color}				  % Controle das cores
\usepackage{graphicx}			  % Inclusão de gráficos
\usepackage{microtype} 		      % para melhorias de justificação
\usepackage{booktabs}
\usepackage{multirow}
\usepackage[table]{xcolor}
\usepackage{subfig}
\usepackage{epstopdf}
\usepackage{hyperref}

\usepackage[mathcal]{eucal}
\usepackage{amsmath}               % great math stuff
\usepackage{amsfonts}              % for blackboard bold, etc
\usepackage{amsthm}                % better theorem environments
\usepackage{amssymb}
\usepackage{mathrsfs}
\DeclareMathAlphabet{\mathpzc}{OT1}{pzc}{m}{it}
\usepackage{undertilde}            % botar tilde embaixo da letra
\usepackage{mathptmx}          % fonte
\usepackage{graphicx}
\graphicspath{{./Figuras/}}    
\definecolor{shadecolor}{rgb}{0.8,0.8,0.8}


%FAZ EDICOES AQUI (somente no conteudo que esta entre entre as ultimas  chaves de cada linha!!!)
\newcommand{\universidade}{Universidade Federal de Pernambuco}
\newcommand{\centro}{Centro Acadêmico do Agreste}
\newcommand{\departamento}{Núcleo de Tecnologia}
\newcommand{\curso}{Engenharia de Produção}
\newcommand{\professor}{Fernando R. L. Contreras}
\newcommand{\disciplina}{Cálculo Diferencial e Integral 3}
%ATE AQUI !!!

\begin{document}
	\pagestyle{empty}
	
	\begin{center}
	%\includegraphics[width=\linewidth/6]{logoUFPE.jpg}%LOGOTIPO DA INSTITUICAO
	 	\vspace{0pt}
	 	
		\universidade
		\par
		\centro
		\par
		\departamento
		\par
		\curso
		\par
		\vspace{08pt}
		\text{ Prova Final - Cálculo Diferencial e Integral 3}\\
		\text{Prof. Fernando R. L. Contreras}	
	\end{center}
	
	%\vspace{0.5pt}
	
	\begin{flushleft}
		Aluno(a):
	\end{flushleft}
	
\begin{itemize}
\item[1.] Calcule $\mathlarger{\int}_{C}e^{x}\cos(y)dx-e^{x}\sin(y)dy$ onde $C$ é qualquer arco de $(1,0)$ a $(0,1)$.
\end{itemize}
\begin{itemize}
\item[2.] Determine o fluxo do rotacional do campo de vetores $F(x,y,z)=(y^{3},x^{3},e^{z})$ através da superficie $S=\left\lbrace (x,y,z)\in R^{3} / x^{2}+y^{2}+z^{2}=2, x^{2}+y^{2}\leq 1, z\geq 0 \right\rbrace $, com normal exterior. 
 \end{itemize}
 \begin{itemize}
\item [3.] Calcule $\mathlarger{\iint}_{S}F.dS$, onde $F(x,y,z)=(4x,-2y^{2},z^{2})$ e $S$ é a superfície limitada por $x^{2}+y^{2}=4$ tal que $0\leq z \leq3$.
\end{itemize}
\begin{itemize}
\item[4.] Seja $R$ a região da elipse $\frac{x^{2}}{4}+\frac{y^{2}}{4}=1$ e exterior a circunferência $x^{2}+y^{2}=1$, calcular a integral de linha $\mathlarger{\int}_{C}2xydx+(x^{2}+2x)dy$ onde $C=C_{1}+C_{2}$ é contorno de $R$.
\end{itemize}

	\flushbottom
	\flushright
     Êxitos...!!!
     \begin{center}
     	%\includegraphics[width=\linewidth/6]{logoUFPE.jpg}%LOGOTIPO DA INSTITUICAO
     	\vspace{0pt}
     	
     	\universidade
     	\par
     	\centro
     	\par
     	\departamento
     	\par
     	\curso
     	\par
     	\vspace{08pt}
     	\text{ Prova Final - Cálculo Diferencial e Integral 3}\\
     	\text{Prof. Fernando R. L. Contreras}	
     \end{center}
     
     \begin{flushleft}
     	Aluno(a):
     \end{flushleft}
 
 \begin{itemize}
 	\item[1.] Calcule $\mathlarger{\int}_{C}e^{x}\cos(y)dx-e^{x}\sin(y)dy$ onde $C$ é qualquer arco de $(1,0)$ a $(0,1)$.
 \end{itemize}
 \begin{itemize}
 	\item[2.] Determine o fluxo do rotacional do campo de vetores $F(x,y,z)=(y^{3},x^{3},e^{z})$ através da superficie $S=\left\lbrace (x,y,z)\in R^{3} / x^{2}+y^{2}+z^{2}=2, x^{2}+y^{2}\leq 1, z\geq 0 \right\rbrace $, com normal exterior. 
 \end{itemize}
 \begin{itemize}
 	\item [3.] Calcule $\mathlarger{\iint}_{S}F.dS$, onde $F(x,y,z)=(4x,-2y^{2},z^{2})$ e $S$ é a superfície limitada por $x^{2}+y^{2}=4$ tal que $0\leq z \leq3$.
 \end{itemize}
 \begin{itemize}
 	\item[4.] Seja $R$ a região da elipse $\frac{x^{2}}{4}+\frac{y^{2}}{4}=1$ e exterior a circunferência $x^{2}+y^{2}=1$, calcular a integral de linha $\mathlarger{\int}_{C}2xydx+(x^{2}+2x)dy$ onde $C=C_{1}+C_{2}$ é contorno de $R$.
 \end{itemize}
\flushbottom
\flushright
Êxitos...!!!
\end{document}

