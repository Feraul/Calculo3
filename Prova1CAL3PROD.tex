\documentclass[oneside,a4paper,12pt]{article}
\usepackage[english,brazilian]{babel}
\usepackage[alf]{abntex2cite}
\usepackage[utf8]{inputenc}
\usepackage[T1]{fontenc}

\usepackage{lastpage}			  % Usado pela Ficha catalográfica
\usepackage{indentfirst}		  % Indenta o primeiro parágrafo de cada
\usepackage[top=20mm, bottom=20mm, left=20mm, right=20mm]{geometry}
\usepackage{framed}
\usepackage{booktabs}

\usepackage{float}
\usepackage{color}				  % Controle das cores
\usepackage{graphicx}			  % Inclusão de gráficos
\usepackage{microtype} 		      % para melhorias de justificação
\usepackage{booktabs}
\usepackage{multirow}
\usepackage[table]{xcolor}
\usepackage{subfig}
\usepackage{epstopdf}
\usepackage{hyperref}

\usepackage[mathcal]{eucal}
\usepackage{amsmath}               % great math stuff
\usepackage{amsfonts}              % for blackboard bold, etc
\usepackage{amsthm}                % better theorem environments
\usepackage{amssymb}
\usepackage{mathrsfs}
\DeclareMathAlphabet{\mathpzc}{OT1}{pzc}{m}{it}
\usepackage{undertilde}            % botar tilde embaixo da letra
\usepackage{mathptmx}          % fonte
\usepackage{graphicx}
\graphicspath{{./Figuras/}}    
\definecolor{shadecolor}{rgb}{0.8,0.8,0.8}


%FAZ EDICOES AQUI (somente no conteudo que esta entre entre as ultimas  chaves de cada linha!!!)
\newcommand{\universidade}{Universidade Federal de Pernambuco}
\newcommand{\centro}{Centro Acadêmico do Agreste}
\newcommand{\departamento}{Núcleo de Tecnologia}
\newcommand{\curso}{Engenharia de Produção}
\newcommand{\professor}{Fernando R. L. Contreras}
\newcommand{\disciplina}{Cálculo Diferencial e Integral 3}
%ATE AQUI !!!

\begin{document}
	\pagestyle{empty}
	
	\begin{center}
	\includegraphics[width=\linewidth/6]{logoUFPE.jpg}%LOGOTIPO DA INSTITUICAO
	 	\vspace{0pt}
	 	
		\universidade
		\par
		\centro
		\par
		\departamento
		\par
		\curso
		\par
		\vspace{08pt}
		\large \textbf{Primeira Prova}
		
	\end{center}
	
	\vspace{1pt}
	
	\begin{tabular}{ |l|p{12cm}| }
		
		\hline
		\multicolumn{2}{|c|}{\textbf{Dados de Identificação}} \\
			\hline
		Disciplina:        &    \disciplina          \\
		\hline
		Professor:         &    \professor           \\
	\hline
	Aluno(a):         &\\
	
		\hline
	\end{tabular}
	
	\vspace{10pt}
\textbf{Justifique todas as suas respostas. Você também será avaliado pela clareza e pela precisão da linguagem utilizada.}
\begin{itemize}
\item[1.] Classificar em condicionalmente convergente, absolutamente convergente ou divergente as séries 
\begin{itemize}
\item[(a)]\textbf{(1.0)}   $\mathlarger{\sum}_{n=2}^{\infty}\frac{1}{n\sqrt[]{n^{2}-1}}$. 
\item[(b)]\textbf{(1.0)} $\sum_{n=1}^{\infty} (-1)^{n+1}\frac{1}{ln(n)} $.\\
\end{itemize}
\end{itemize}
\begin{itemize} 
\item[2.]\textbf{(2.0)} Considere a série de potência $f(x)=\mathlarger{\sum}_{n=2}^{\infty}\frac{x^{n}}{(n-1)n}$.
	\begin{itemize} 
	\item[(a)] Determine o raio de convergência e estudar o comportamento da série nos valores extremos do intervalo aberto da convergência absoluta.
	\item[(b)] Calcular a função $f(x)$ que representa a série no intervalo aberto de convergência absoluta.
	\textit{Sugestão. Usar os resultados de derivação e integração termo a termo e lembrar que uma primitiva da função} $g(t)=ln(t)$ é dada por $G(t)=tln(t)-t$.
    \end{itemize}
\end{itemize}
\begin{itemize}
\item[3.] Mostre as seguintes sequências
	\begin{itemize}
	\item[(a)]\textbf{(1.0)} $\lim\limits_{n\longrightarrow \infty} \sqrt[n]{a^{n}+b^{n}}=b$, $0<a<b$.
	\item[(b)]\textbf{(1.0)} Seja $\left\lbrace a_{n}\right\rbrace $ tal que $a_{n+1}=\frac{3(1+a_{n})}{3+a_{n}}$, $a_{1}=3$. Mostre que $\left\lbrace a_{n}\right\rbrace $ tende a\quad $\sqrt[]{3}$ .
    \end{itemize}
\end{itemize}
\begin{itemize}
\item[4.]\textbf{(2.0)} Considere a função $f(x,y)=\frac{y^{3}}{3}-9y+x^{2}$.
\begin{itemize} 
	\item[(a)] Determine os pontos críticos da função $f$ em todo $\mathbb{R}$.
	\item[(b)] Determine os pontos máximo e mínimo de $f$ sobre a circunferência $x^{2}+(y+3)^{2}=9$.
	\item[(c)] Determine os valores máximo e mínimo de $f$ no disco $D=\left\lbrace (x,y):\quad x^{2}+(y+3)^{2}\leq 9\right\rbrace $.
\end{itemize}
\end{itemize}
\begin{itemize}
\item[5.]\textbf{(2.0)} Sejam $f(x,y)=x^{3}+y^{3}-x^{2}+4y$ e $P_{1}(x,y)$ o polinômio de Taylor de ordem 1 de $f$ em volta de $(1,1)$. Mostre que para todo $(x,y)$, com $|x-1|<1$ e $|y-1|<1$, $\left|f(x,y)-P_{1}(x,y)\right|<7(x-1)^{2}+6(y-1)^{2}$. 
\end{itemize}
	\flushbottom
	\flushright
     Êxitos...!!!
\end{document}
