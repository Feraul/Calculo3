\documentclass[oneside,a4paper,12pt]{article}
\usepackage[english,brazilian]{babel}
\usepackage[alf]{abntex2cite}
\usepackage[utf8]{inputenc}
\usepackage[T1]{fontenc}

\usepackage{lastpage}			  % Usado pela Ficha catalográfica
\usepackage{indentfirst}		  % Indenta o primeiro parágrafo de cada
\usepackage[top=20mm, bottom=20mm, left=20mm, right=20mm]{geometry}
\usepackage{framed}
\usepackage{booktabs}

\usepackage{float}
\usepackage{color}				  % Controle das cores
\usepackage{graphicx}			  % Inclusão de gráficos
\usepackage{microtype} 		      % para melhorias de justificação
\usepackage{booktabs}
\usepackage{multirow}
\usepackage[table]{xcolor}
\usepackage{subfig}
\usepackage{epstopdf}
\usepackage{hyperref}

\usepackage[mathcal]{eucal}
\usepackage{amsmath}               % great math stuff
\usepackage{amsfonts}              % for blackboard bold, etc
\usepackage{amsthm}                % better theorem environments
\usepackage{amssymb}
\usepackage{mathrsfs}
\DeclareMathAlphabet{\mathpzc}{OT1}{pzc}{m}{it}
\usepackage{undertilde}            % botar tilde embaixo da letra
\usepackage{mathptmx}          % fonte
\usepackage{graphicx}
\graphicspath{{./Figuras/}}    
\definecolor{shadecolor}{rgb}{0.8,0.8,0.8}


%FAZ EDICOES AQUI (somente no conteudo que esta entre entre as ultimas  chaves de cada linha!!!)
\newcommand{\universidade}{Universidade Federal de Pernambuco}
\newcommand{\centro}{Centro Acadêmico do Agreste}
\newcommand{\departamento}{Núcleo de Tecnologia}
\newcommand{\curso}{Engenharia de Produção}
\newcommand{\professor}{Fernando R. L. Contreras}
\newcommand{\disciplina}{Cálculo Diferencial e Integral 3}
%ATE AQUI !!!

\begin{document}
	\pagestyle{empty}
	
	\begin{center}
	%\includegraphics[width=\linewidth/6]{logoUFPE.jpg}%LOGOTIPO DA INSTITUICAO
	 	\vspace{0pt}
	 	
		\universidade
		\par
		\centro
		\par
		\departamento
		\par
		\curso
		\par
		\vspace{08pt}
		\text{Segunda Chamada}\\ \text{ Prova 3 - Cálculo Diferencial e Integral 3}\\
		\text{Prof. Fernando R. L. Contreras}	
	\end{center}
	
	%\vspace{0.5pt}
	
	\begin{flushleft}
		Aluno(a):
	\end{flushleft}
	
\begin{itemize}
\item[1.] Verifique o Teorema de Green para a integral $\mathlarger{\oint}_{C} (2x^{3}-y^{3})dx+(x^{3}+y^{3})dy$ sobre a curva $C$ dada por $x^{2}+y^{2}=1$ .
\end{itemize}
\begin{itemize}
\item[2.] Seja $W$ a região sólida limitada pelos planos coordenados e o plano $2x+2y+z=6$ e seja $F(x,y,z)=(x,y^{2},z)$, calcular $\mathlarger{\iint}_{S}F\cdot dS$, onde $S$ é a superfície de $W$. 
 \end{itemize}
 \begin{itemize}
\item [3.] Calcule $\mathlarger{\iint}_{S} \nabla \times F\cdot dS$ onde $S=\{(x,y,z)\in R^{3}/ x=-1+y^{2}+z^{2}, x\leq 0 \}$ e o campo $F$ é definido por $F(x,y,z)=(xz,ze^{x},-y)$.
\end{itemize}
\begin{itemize}
\item[4.]Seja $R$ a região da elipse $\frac{x^{2}}{9}+\frac{y^{2}}{4}=1$ e exterior a circunferência $x^{2}+y^{2}=1$, calcular a integral de linha $\mathlarger{\int}_{C}2xydx+(x^{2}+2x)dy$ onde $C=C_{1}+C_{2}$ é contorno de $R$. .
\end{itemize}
\begin{itemize}
	\item[Opcional.] Descreva o Teorema de Stokes.
\end{itemize}
     \begin{center}
     	%\includegraphics[width=\linewidth/6]{logoUFPE.jpg}%LOGOTIPO DA INSTITUICAO
     	\vspace{0pt}
     	
     	\universidade
     	\par
     	\centro
     	\par
     	\departamento
     	\par
     	\curso
     	\par
     	\vspace{08pt}
     	\text{Segunda Chamada }\\ \text{ Prova 1 - Cálculo Diferencial e Integral 3}\\
     	\text{Prof. Fernando R. L. Contreras}	
     \end{center}
     
     \begin{flushleft}
     	Aluno(a):
     \end{flushleft}
 
 \begin{itemize}
 	\item[1.]  Encontre o raio de convergência e o intervalo de convergência da série $\mathlarger{\sum}_{n=0}^{\infty} \frac{n(x+2)^{n}}{3^{n+1}}$.
 \end{itemize}
 \begin{itemize}
 	\item[2.] Investigue a sequência ${a_{n}}$ definida pela relação de recorrência $a_{1}=2$ e $a_{n+1}=\frac{1}{2}(a_{n}+6)$ para $n=1,2,...$  . 
 \end{itemize}
 \begin{itemize}
 	\item [3.] Determine o polinômio de Taylor de ordem 2 da função dada, em volta do ponto $(1,1)$ dado por  $f(x,y)=x^{3}+2x^{2}y+3y^{3}+x-y$. 
 \end{itemize}
 \begin{itemize}
 	\item[4.] Sejam $f(x,y)=x^{3}+y^{3}-x^{2}+4y$ e $P_{1}(x,y)$ o polinômio de Taylor de ordem 1 de $f$ em volta de $(1,1)$. Mostre que para todo $(x,y)$, com $|x-1|<1$ e $|y-1|<1$, $\left|f(x,y)-P_{1}(x,y)\right|<7(x-1)^{2}+6(y-1)^{2}$..
 \end{itemize}
\begin{itemize}
	\item[Opcional.] Escreva a definição de limite de sequência na forma simbólica.
\end{itemize}
\end{document}

