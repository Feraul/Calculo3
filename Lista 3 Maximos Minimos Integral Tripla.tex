
%%%%%%%%%%%%%%%%%%%%%%%%%%%%%%%%%%%%%%%%%%%%%%%%%%%%%%%%%%%%%%%%%%%%%%%%
%    Option test file, will be created during the first LaTeX run:
\begin{filecontents}{exercise.thm}
\def\th@exercise{%
  \normalfont % body font
  \thm@headpunct{:}%
}
\end{filecontents}
%%%%%%%%%%%%%%%%%%%%%%%%%%%%%%%%%%%%%%%%%%%%%%%%%%%%%%%%%%%%%%%%%%%%%%%%

\documentclass[12pt,openright,oneside,a4paper,english,french,spanish,brazil]{article}
% ---
% Pacotes básicos 
% ---
\usepackage{lmodern}			  % Usa a fonte Latin Modern			
\usepackage[T1]{fontenc}		  % Selecao de codigos de fonte.
\usepackage[utf8]{inputenc}	      % Codificacao do documento (conversão automática dos acentos)
\usepackage[top=20mm, bottom=20mm, left=20mm, right=20mm]{geometry}
\usepackage{lastpage}			  % Usado pela Ficha catalográfica
\usepackage{indentfirst}		  % Indenta o primeiro parágrafo de cada seção.
\usepackage{color}				  % Controle das cores
\usepackage{graphicx}			  % Inclusão de gráficos
\usepackage{microtype} 		      % para melhorias de justificação
\usepackage{booktabs}
\usepackage{multirow}
\usepackage[table]{xcolor}
\usepackage{subfig}
\usepackage{epstopdf}
\usepackage{hyperref}
\usepackage[mathcal]{eucal}
\usepackage{amsmath}               % great math stuff
\usepackage{amsfonts}              % for blackboard bold, etc
\usepackage{amsthm}                % better theorem environments
\usepackage{amssymb}
\usepackage{mathrsfs}
\DeclareMathAlphabet{\mathpzc}{OT1}{pzc}{m}{it}
\usepackage{undertilde}            % botar tilde embaixo da letra
\usepackage{mathptmx}          % fonte
\usepackage{latexsym}
\usepackage{makeidx}            % para definir o índice
\usepackage{epsfig}             % para introduzir figuras no formato eps
\usepackage{bbm}
\usepackage{dsfont}
\usepackage{graphicx,color}     % permite a inclusao de figuras
\usepackage{verbatim}
\usepackage{gensymb}
\usepackage{titling}
\newcommand{\subtitle}[1]{%
	\posttitle{%
		\par\end{center}
	\begin{center}\Large#1\end{center}
	\vskip0.5em}%
}



\newtheorem{df}{Definição}
\newtheorem{ex}{Exemplo}
\newtheorem{teo}{Teorema}

\newtheoremstyle{note}% name
  {3pt}%      Space above
  {3pt}%      Space below
  {}%         Body font
  {}%         Indent amount (empty = no indent, \parindent = para indent)
  {\itshape}% Thm head font
  {:}%        Punctuation after thm head
  {.5em}%     Space after thm head: " " = normal interword space;
        %       \newline = linebreak
  {}%         Thm head spec (can be left empty, meaning `normal')

\theoremstyle{note}
\newtheorem{note}{Note}

\newtheoremstyle{citing}% name
  {3pt}%      Space above, empty = `usual value'
  {3pt}%      Space below
  {\itshape}% Body font
  {}%         Indent amount (empty = no indent, \parindent = para indent)
  {\bfseries}% Thm head font
  {.}%        Punctuation after thm head
  {.5em}%     Space after thm head: " " = normal interword space;
        %       \newline = linebreak
  {\thmnote{#3}}% Thm head spec

\theoremstyle{citing}
\newtheorem*{varthm}{}% all text supplied in the note

\newtheoremstyle{break}% name
  {9pt}%      Space above, empty = `usual value'
  {9pt}%      Space below
  {\itshape}% Body font
  {}%         Indent amount (empty = no indent, \parindent = para indent)
  {\bfseries}% Thm head font
  {.}%        Punctuation after thm head
  {\newline}% Space after thm head: \newline = linebreak
  {}%         Thm head spec

\theoremstyle{break}
\newtheorem{bthm}{B-Theorem}

\theoremstyle{exercise}
\newtheorem{exer}{Exercise}

\swapnumbers
\theoremstyle{plain}
\newtheorem{thmsw}{Theorem}[section]
%\newtheorem{corsw}[thm]{Corollary}
\newtheorem{propsw}{Proposition}
%\newtheorem{lemsw}[thm]{Lemma}

%    Because the amsmath pkg is not used, we need to define a couple of
%    commands in more primitive terms.
\let\lvert=|\let\rvert=|
\newcommand{\Ric}{\mathop{\mathrm{Ric}}\nolimits}

%    Dispel annoying problem of slightly overlong lines:
\addtolength{\textwidth}{8pt}

\title{ \textbf{Notas de Aula}}

\author{\textbf{Fernando Contreras}\\
	\large Nucleo de Tecnologia\\
	Universidade Federal de Pernambuco (UFPE)}



\begin{document}
	\begin{center}
		Universidade Federal de Pernambuco (UFPE)\\
		Centro Acadêmico do Agreste\\
		Núcleo de Tecnologia\\
		
		Lista 3 de Calculo Diferencial e Integral 3\\
		Prof. Fernando RL Contreras
	\end{center}


Sejam os seguintes problemas relativos a Maximos, Minimos e Integrais triplas.

%\begin{multicols}{2}

\begin{itemize}
	\item[1.] Determine os valoress máximos e mínimos locais e pontos de sela da função. 
	\begin{itemize}
	\item[a.] $f(x,y)=xy+\frac{1}{x}+\frac{1}{y}$
	problema 12 pag. 885, stewart	
	\item[b.] $f(x,y)=(x^{2}+y^{2})e^{y^{2}-x^{2}}$
	problema 15 pag. 885, stewart	
    \item[c.] $f(x,y)=\sin (x) \sin(y)$, $-\pi<x<\pi$, $-\pi<y<\pi$
	problema 18 pag. 885, stewart
    \end{itemize}	
\end{itemize}
\begin{itemize}
	\item[2.] Determine os valores máximo e mínimo absoluto de $f$ no conjunto $D$.   
	\begin{itemize}
		\item[a.] $f(x,y)=1+4x-5y$, D é região triangular fechada com vértices $(0,0), (2,0)$ e $(0.3)$
		problema 29 pag. 885, stewart	
		\item[b.] $f(x,y)=x^{2}+y^{2}+x^{2}y+4$, $D=\{(x,y)| \quad |x|\leqslant 1, |y|\leqslant 1 \}$
		problema 31 pag. 885, stewart
		\item[c.] $f(x,y)=xy^{2}$, $D=\{(x,y)| \quad 0\leqslant x, 0\leqslant y, x^{2}+y^{2}\leqslant3 \}$
		problema 34 pag. 885, stewart		
	\end{itemize}
\end{itemize}
\begin{itemize}
	\item[3.] Utilize os multiplicadores de Lagrange para determinar os valores maximos e mínimos da função sujeita à(s) restrição(ões) dada(s).
	\begin{itemize}
	\item[a.] $f(x,y)=x^{2}+y^{2}$; $xy=1$
	problema 3 pag. 893, stewart	
	\item[b.] $f(x,y,z,t)=x+y+z+t$; $x^{2}+y^{2}+z^{2}+t^{2}=1$
	problema 13 pag. 893, stewart	
	\item[c.] $f(x,y,z)=x+2y$; $x+y+z=1$ e $y^{2}+z^{2}=4$
	problema 15 pag. 893, stewart	
	\item[d.] $f(x,y,z)=yz+xy$; $xy=1$ e $y^{2}+z^{2}=1$
	problema 17 pag. 893, stewart
    \end{itemize}
\end{itemize}
\begin{itemize}
	\item[4.] Determine os valores extremos de $f$ na região descrita pela desigualdade.
	\begin{itemize}
		\item[a.] $f(x,y)=2x^{2}+3y^{2}-4x-5$; $x^{2}+y^{2}\leqslant 16$
		problema 18 pag. 894, stewart	
		\item[b.] $f(x,y)=e^{-xy}$; $x^{2}+4y^{2}\leqslant 1$
     	problema 19 pag. 894, stewart		
	\end{itemize}
\end{itemize}

\begin{itemize}
	\item[5.] Uma caixa retangular sem tampa deverá ter um volume fixo. Como deverá ser feito a caixa para empregar em sua fabricação a menor quantidade de material?
	problema 12 pag. 515, Eduardo Espinosa	
\end{itemize}
\begin{itemize}
	\item[6.] Suponha que $T(x,y)=4-x^{2}-y^{2}$ represente uma distribuição de temperatura no plano. Seja $A=\{(x,y)\in \mathbbm{R}^{2} | x\geqslant 0, y\geqslant x, 2y+x \leqslant 4\} $. Determine o ponto de A de menor temperatura 
		problema 3 pag.322, Gudorizzi vol 2	
\end{itemize}
\begin{itemize}
	\item[7.] A temperatura $T$ em qualquer ponto $(x,y,z)$ do espaço é dada por $T=100x^{2}yz$. Determine a temperatura máxima sobre a esfera $x^{2}+y^{2}+z^{2}\leqslant 4$. Qual a temperatura mínima?
		problema 23 pag.333, Gudorizzi vol 2 
\end{itemize}
\begin{itemize}
	\item[8.] Avalie a integral de $f(x,y,z)=3z+xz$, sobre o sólido $E$ limitado pelo cilindro $x^{2}+z^{2}=9$ e pelos planos $x+y=3, z=0, y=0$ sobre o plano $XY$. Rpta: $\frac{648}{5}$
	 problema 3.4 pag. 192 Venero
\end{itemize}
\begin{itemize}
	\item[9.]Seja $E$ o sólido interior ao cilindro $y^{2}+z^{2}=4$, limitado pelas superficies cilindricas $x=z^{2}$, $x-6=(z-2)^{2}$. Calcule o volume de $E$. Rpta: $40 \pi$
	 problema 3.8 pag. 194 Venero
\end{itemize}
\begin{itemize}
	\item[10.] Calcule a integral tripla de $f(x,y,z)=\sqrt{4-z}$ sobre o sólido $E$ limitado pelos cilindros $y^{2}=2x$, $y^{2}=8-2x$, $y^{2}=4-z$.Rpta: $\frac{448}{9}$
	 problema 3.11 pag. 197 Venero
\end{itemize}
\begin{itemize}
	\item [11.] Ache o volume de um porção $E$ da esfera $x^{2}+y^{2}+z^{2}\leqslant a^{2}$, $a>0$ que esta no interior do cilindro $r=a\sin \theta$. Rpta: $\frac{4a^{3}}{3}(\frac{\pi}{2}-\frac{2}{3})$
		 problema 5.2 pag. 208 Venero
\end{itemize}
\begin{itemize}
	\item [12.] Calcule a integral da função $f(x,y,z)=\sqrt{9-x^{2}-2y^{2}}$ sobre o sólido $S$limitado superiormente pelo paraboloide $z=9-x^{2}-2y^{2}$ e inferiormente pelo plano $z=5$. Rpta: $5\sqrt{10}/3 +9\sqrt{2}/5$
	 problema 5.12 pag. 215 Venero 
\end{itemize}
\begin{itemize}
	\item [13.] Ache o volume do sólido $S$ exterior a: $y^{2}=x^{2}+z^{2}$ e interior a  esfera $x^{2}+y^{2}+z^{2}=2ax$, $a>0$. Rpta:  $\frac{2a^{3}}{9}(3\pi +8)$ 
	 problema 6.9 pag. 234 Venero 
\end{itemize}
\begin{itemize}
	\item [14.]  Encontrar a massa do sólido esférico de radio "a" se a densidade de volume em qualquer ponto é proporcional a distancia do ponto ao centro da esfera. Rpta: $\pi ka^{3}$, $k$ constante de proporcionalidade. 
	 problema 12 pag. 734 Eduardo Espinoza
\end{itemize}
\begin{itemize}
	\item [15.] Encontrar o momento de inercia com respeito ao eixo $Z$ do sólido homogêneo dentro do paraboloide $x^{2}+y^{2}=z$ e fora do cone $x^{2}+y^{2}=z^{2}$, $p$ é a densidade de volume constante $k$.   Rpta: $\frac{k\pi}{15}$
	 problema 13 pag. 734 Eduardo Espinoza
\end{itemize}
\begin{itemize}
	\item [16.] Calcule o centro de massa do corpo homogêneo $x^{2}+y^{2}\leqslant z \leqslant 1$. Rpta: $(0,0,\frac{2}{3})$
	 problema 2 pag. 138 Guidorizzi vol 3 
\end{itemize}
\begin{itemize}
	\item [17.] Calcule momento de inercia de uma esfera homogênea, de raio $R$, em relação ao eixo passando pelo seu centro. Rpta: $\frac{2}{3}MR^{2}$, onde $M=\frac{4}{3}\pi R^{3}k$ é massa da esfera e $k$ é a densidade constante.
	 problema 1 pag. 137 Guidorizzi vol 3 
\end{itemize}

\begin{itemize}
	\item [18.] Calcule  $\iiint\limits_B\frac{\sin (x+y-z)}{x+2y+z}dxdydz$ onde $B$ é o paralelepípedo $1\leqslant x+2y+z\leqslant 2$, $0\leqslant x+y-z\leqslant \frac{\pi}{4}$ e $0\leqslant z \leqslant 1$.Rpta: $\ln(2)(1-\frac{\sqrt{2}}{2})$
	 problema 1 pag. 117 Guidorizzi vol 3 
\end{itemize}
\begin{itemize}
	\item [19.] Calcule o volume do conjunto de todos $(x,y,z)$ tais que $1\leqslant x+y+z\leqslant 3$, $x+y\leqslant z \leqslant x+y+2$, $x\geqslant 0 $ e $y\geqslant 0$. Rpta: $\frac{24}{25}$
	 problema 6 pag. 121 Guidorizzi vol 3 
\end{itemize}
\begin{itemize}
\item [20.] Calcule a massa do cilindro $x^{2}+y^{2}\leqslant 1$, $0\leqslant z \leqslant 1$ admitindo que a densidade seja dada por $x^{2}$. Rpta: $\frac{\pi}{4} $ 
 problema 3 pag. 110 Guidorizzi vol 3 
\end{itemize}
\end{document}
